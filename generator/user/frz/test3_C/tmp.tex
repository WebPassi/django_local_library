\documentclass[fleqn,leqno,12pt]{scrartcl} % Formel links ausgerichtet und nummeriert
%\documentclass[a4paper,12pt]{article}
%\usepackage{a4wide}
\usepackage{pdflscape}
\usepackage{comment}
\usepackage[ngerman]{babel}
\usepackage[utf8]{inputenc}
\usepackage{pgfplots}
\usetikzlibrary{arrows}
\usepackage {graphicx}
\usepackage{paralist} % enumerate
\usepackage{multicol}

\usepackage{fancyheadings}
%\pagestyle{empty}
\usepackage{amsthm}   % \newtheorem-Umgebung
\usepackage{amsmath}  % \align-Umgebung
\usepackage{amssymb}  % z.B. fuer reelles Zahlensymbol
\usepackage{framed}

\usepackage[left=1.5cm,right=1.5cm,top=1.5cm,bottom=0.5cm]{geometry} 
% environment styles
\newtheoremstyle{note}
{16pt}  % Platz zwischen Kopf und voherigem Text
{16pt}  % und nachfolgendem Text
{}     % Schriftart des Koerpers
{}     % mit \parindent Einzug
{\bf}  % Schriftart des Kopfes
{:}     % Nach Bedarf z.B. Doppelpunkt nach dem Kopf
{0.5em} % Platz zwischen Kopf und Koerper
{}     % Kopfname

\newtheoremstyle{note2}
{40pt}  % Platz zwischen Kopf und voherigem Text
{16pt}  % und nachfolgendem Text
{}     % Schriftart des Koerpers
{}     % mit \parindent Einzug
{\bf}  % Schriftart des Kopfes
{:}     % Nach Bedarf z.B. Doppelpunkt nach dem Kopf
{0.5em} % Platz zwischen Kopf und Koerper
{}     % Kopfname

\theoremstyle{note2}
\newtheorem{aufgabe}{Aufgabe}
\newcommand{\Vekz}[2]{\left(\begin{array}{r} #1 \\ #2 \end{array}\right)}
\newcommand{\Vekd}[3]{\left(\begin{array}{r} #1 \\ #2 \\ #3 \end{array}\right)}


\begin{document} 
\begin{aufgabe} ~ \\ 
Ordne die folgenden Zahlen der Größe nach. Beginne mit der größten Zahl.\[-5.26\quad ; \quad-4.4\quad ; \quad8.9\quad ; \quad-6.2\quad ; \quad-4.25\quad ; \quad\]\underline{\hspace{12cm}}\end{aufgabe} 
 
\begin{aufgabe} ~ \\ 
Löse die Klammern auf.\begin{multicols}{2} 
\begin{enumerate}[a)] 
\item 
$4\cdot(2x-2)$=
\item 
$3\cdot(8x+2)$=
\end{enumerate} 
\end{multicols} 
\end{aufgabe} 
 
\begin{aufgabe}\framebox{\qquad/24} ~ \\ 
Bestimmen Sie die Wahrscheinlichkeiten der Bernoulli-Experimente mit den folgenden Kenngrößen und Trefferanzahl $k$.\begin{multicols}{2} 
\begin{enumerate}[a)] 
\item 
$n=27$, $p=0.3$, $k=8$
\item 
$n=80$, $p=\frac{1}{3}$, $22\leq k \leq 30$
\item 
$n=80$, $p=0.25$, $k$ mindestens $20$
\item 
$n=50$, $p=\frac{1}{3}$, $13\leq k \leq 20$
\item 
$n=50$, $p=0.25$, $9\leq k \leq 15$
\item 
$n=50$, $p=0.1$, $5\leq k $
\end{enumerate} 
\end{multicols} 
\end{aufgabe} 
 
\begin{aufgabe} ~ \\ 
Geben Sie zur folgenden Ebene eine Koordinatenform an.\[E: (\vec{x} - \Vekd{4}{4}{-3}) \cdot \Vekd{-5}{-3}{-3} = 0 \]\end{aufgabe} 
 
\end{document} 
