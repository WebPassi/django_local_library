\documentclass[fleqn,leqno,12pt]{scrartcl} % Formel links ausgerichtet und nummeriert
%\documentclass[a4paper,12pt]{article}
%\usepackage{a4wide}
\usepackage{pdflscape}
\usepackage{comment}
\usepackage[ngerman]{babel}
\usepackage[utf8]{inputenc}
\usepackage{pgfplots}
\usetikzlibrary{arrows}
\usepackage {graphicx}
\usepackage{paralist} % enumerate
\usepackage{multicol}

\usepackage{fancyheadings}
%\pagestyle{empty}
\usepackage{amsthm}   % \newtheorem-Umgebung
\usepackage{amsmath}  % \align-Umgebung
\usepackage{amssymb}  % z.B. fuer reelles Zahlensymbol
\usepackage{framed}

\usepackage[left=1.5cm,right=1.5cm,top=1.5cm,bottom=0.5cm]{geometry} 
% environment styles
\newtheoremstyle{note}
{16pt}  % Platz zwischen Kopf und voherigem Text
{16pt}  % und nachfolgendem Text
{}     % Schriftart des Koerpers
{}     % mit \parindent Einzug
{\bf}  % Schriftart des Kopfes
{:}     % Nach Bedarf z.B. Doppelpunkt nach dem Kopf
{0.5em} % Platz zwischen Kopf und Koerper
{}     % Kopfname

\newtheoremstyle{note2}
{40pt}  % Platz zwischen Kopf und voherigem Text
{16pt}  % und nachfolgendem Text
{}     % Schriftart des Koerpers
{}     % mit \parindent Einzug
{\bf}  % Schriftart des Kopfes
{:}     % Nach Bedarf z.B. Doppelpunkt nach dem Kopf
{0.5em} % Platz zwischen Kopf und Koerper
{}     % Kopfname

\theoremstyle{note2}
\newtheorem{aufgabe}{Aufgabe}
\newcommand{\Vekz}[2]{\left(\begin{array}{r} #1 \\ #2 \end{array}\right)}
\newcommand{\Vekd}[3]{\left(\begin{array}{r} #1 \\ #2 \\ #3 \end{array}\right)}


\begin{document} 
\begin{aufgabe} ~ \\ 
Ordne die folgenden Zahlen der Gr\"o\ss{}e nach. Beginne mit der kleinsten Zahl.\[-5.26\quad ; \quad-4.4\quad ; \quad8.9\quad ; \quad-6.2\quad ; \quad-4.25\quad ; \quad\]\end{aufgabe} 
 
\begin{aufgabe}\framebox{\qquad/24} ~ \\ 
Bestimmen Sie die Wahrscheinlichkeiten der Bernoulli-Experimente mit den folgenden Kenngrößen und Trefferanzahl $k$.\begin{multicols}{2} 
\begin{enumerate}[a)] 
\item 
$n=24$, $p=0.6$, $k=14$\\ 
$P(X=14)=\binom{24}{14}\cdot0.6^{14}\cdot0.4^{10}=16.12$
\item 
$n=50$, $p=0.25$, $k$ mindestens $12$\\ 
$P(12\leq X)=$ $61.84$
\item 
$n=20$, $p=0.25$, $k \leq 5$\\ 
$P(X \leq 5)=$ $61.72$
\item 
$n=80$, $p=\frac{1}{6}$, $k$ höchstens $13$\\ 
$P(X \leq 13)=$ $53.33$
\item 
$n=80$, $p=0.5$, $35\leq k \leq 44$\\ 
$P(35\leq X \leq 44)=$ $73.36$
\item 
$n=80$, $p=0.5$, $40\leq k $\\ 
$P(40\leq X)=$ $54.45$
\end{enumerate} 
\end{multicols} 
\end{aufgabe} 
 
\begin{aufgabe} ~ \\ 
Ordne die folgenden Zahlen der Gr\"o\ss{}e nach. Beginne mit der kleinsten Zahl.\[-8.1\quad ; \quad-8.5\quad ; \quad4.7\quad ; \quad7.5\quad ; \quad-0.8\quad ; \quad\]\end{aufgabe} 
 
\begin{aufgabe} ~ \\ 
Geben Sie zur folgenden Ebene eine Koordinatenform an.\[E: 2x_1 + 4x_2 + 5x_3 = -14\]\end{aufgabe} 
 
\end{document} 
