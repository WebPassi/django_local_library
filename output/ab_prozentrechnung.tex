\documentclass[12pt]{article}
\usepackage[utf8]{inputenc}
\usepackage{paralist} 
\usepackage{amssymb}  
\usepackage{amsthm}   
\usepackage{eurosym} 
\usepackage{multicol} 
\usepackage[left=1.5cm,right=1.5cm,top=1.5cm,bottom=0.5cm]{geometry} 
\usepackage{fancyheadings} 
\pagestyle{fancy} 
\headheight1.6cm 
\lhead{} 
\chead{Prozentrechnung} 
\rhead{\includegraphics[scale=0.5]{/home/pfranz/hbo/logo.png}} 
\lfoot{} 
\cfoot{} 
\rfoot{} 
\usepackage{amsmath}  
\usepackage{cancel} 
\usepackage{pgf,tikz} 
\usetikzlibrary{arrows} 
\newtheoremstyle{aufg} 
{16pt}  % Platz zwischen Kopf und voherigem Text 
{16pt}  % und nachfolgendem Text 
{}     % Schriftart des Koerpers 
{}     % mit \parindent Einzug 
{\bf}  % Schriftart des Kopfes 
{:}     % Nach Bedarf z.B. Doppelpunkt nach dem Kopf 
{0.5em} % Platz zwischen Kopf und Koerper 
{}     % Kopfname 

\theoremstyle{aufg} 
\newtheorem{aufgabe}{Aufgabe} 



\newtheoremstyle{bsp} 
{16pt}  % Platz zwischen Kopf und voherigem Text 
{16pt}  % und nachfolgendem Text 
{}     % Schriftart des Koerpers 
{}     % mit \parindent Einzug 
{\em}  % Schriftart des Kopfes 
{:}     % Nach Bedarf z.B. Doppelpunkt nach dem Kopf 
{0.5em} % Platz zwischen Kopf und Koerper 
{}     % Kopfname 

\theoremstyle{bsp} 
\newtheorem{beispiel}{Beispiel} 



\begin{document} 
    \begin{flushleft}
Name: \hspace{12cm} Datum:\begin{beispiel} ~ \\ 
Wieviel sind $6\%$ von $5500$ \euro?\qquad {\em L\"osung:} $\frac{6}{100}\cdot 5500$ \euro $ = 330,\hspace{-1pt}00$ \euro\end{beispiel} 
\begin{aufgabe} ~ \\ 
Bestimme den Prozentwert.\begin{multicols}{3} 
\begin{enumerate}[a)] 
\item 
$27\%$ von $2843$ g
\item 
$85\%$ von $1012$ kg
\item 
$46\%$ von $1948$ m
\item 
$10\%$ von $3769$ km
\item 
$63\%$ von $329$ ml
\item 
$67\%$ von $2520$ t
\item 
$89\%$ von $4032$ \euro~
\item 
$62\%$ von $626$ cm$^2$
\item 
$79\%$ von $3002$ m$^3$
\end{enumerate} 
\end{multicols} 
\end{aufgabe} 
\begin{beispiel} ~ \\ 
Wieviel Prozent sind $2100$ \euro~von $7500$ \euro?\qquad {\em L\"osung:} $\frac{2100}{7500}\cdot 100\%  = 28,\hspace{-1pt}000\%$ \end{beispiel} 
\begin{aufgabe} ~ \\ 
Bestimme den Prozentsatz.\begin{multicols}{3} 
\begin{enumerate}[a)] 
\item 
$1557$ g von $3158$ g
\item 
$41$ kg von $3498$ kg
\item 
$877$ m von $3486$ m
\item 
$2188$ km von $4691$ km
\item 
$1468$ ml von $3190$ ml
\item 
$2227$ t von $3940$ t
\item 
$957$ \euro~ von $4126$ \euro~
\item 
$958$ cm$^2$ von $3823$ cm$^2$
\item 
$2607$ m$^3$ von $4752$ m$^3$
\end{enumerate} 
\end{multicols} 
\end{aufgabe} 
\begin{beispiel} ~ \\ 
$2100$ sind $45\%$. Wieviel sind 100\%?\qquad {\em L\"osung:} $2100$ \euro$\cdot \frac{100}{45}= 4666,\hspace{-1pt}7$ \euro\end{beispiel} 
\begin{aufgabe} ~ \\ 
Bestimme den Grundwert.\begin{multicols}{3} 
\begin{enumerate}[a)] 
\item 
$1126$ g sind $18\%$.
\item 
$4958$ kg sind $23\%$.
\item 
$4813$ m sind $24\%$.
\item 
$4385$ km sind $20\%$.
\item 
$3249$ ml sind $27\%$.
\item 
$2902$ t sind $28\%$.
\item 
$2977$ \euro~ sind $58\%$.
\item 
$1469$ cm$^2$ sind $45\%$.
\item 
$1236$ m$^3$ sind $67\%$.
\end{enumerate} 
\end{multicols} 
\end{aufgabe} 
$ \scriptstyle25,\hspace{-1pt}0589$ | $ \scriptstyle12033,\hspace{-1pt}3$ | $ \scriptstyle49,\hspace{-1pt}3034$ | $ \scriptstyle56,\hspace{-1pt}5228$ | $ \scriptstyle2371,\hspace{-1pt}58$ | $ \scriptstyle767,\hspace{-1pt}610$ | $ \scriptstyle860,\hspace{-1pt}200$ | $ \scriptstyle207,\hspace{-1pt}270$ | $ \scriptstyle1,\hspace{-1pt}17210$ | $ \scriptstyle21556,\hspace{-1pt}5$ | $ \scriptstyle3264,\hspace{-1pt}44$ | $ \scriptstyle10364,\hspace{-1pt}3$ | $ \scriptstyle21925,\hspace{-1pt}0$ | $ \scriptstyle46,\hspace{-1pt}6425$ | $ \scriptstyle5132,\hspace{-1pt}76$ | $ \scriptstyle1688,\hspace{-1pt}40$ | $ \scriptstyle3588,\hspace{-1pt}48$ | $ \scriptstyle23,\hspace{-1pt}1944$ | $ \scriptstyle376,\hspace{-1pt}900$ | $ \scriptstyle6255,\hspace{-1pt}56$ | $ \scriptstyle20054,\hspace{-1pt}2$ | $ \scriptstyle46,\hspace{-1pt}0188$ | $ \scriptstyle25,\hspace{-1pt}1578$ | $ \scriptstyle896,\hspace{-1pt}080$ | $ \scriptstyle54,\hspace{-1pt}8611$ | $ \scriptstyle388,\hspace{-1pt}120$ | $ \scriptstyle1844,\hspace{-1pt}78$ | \end{flushleft} 
    \end{document}