\documentclass[12pt]{article}
\usepackage[utf8]{inputenc}
\usepackage{paralist} 
\usepackage{amssymb}  
\usepackage{amsthm}   
\usepackage{multicol} 
\usepackage[left=1.5cm,right=1.5cm,top=1.5cm,bottom=0.5cm]{geometry} 
\usepackage{fancyheadings} 
\pagestyle{fancy} 
\lhead{} 
\chead{Parabeln und Potenzfunktionen \\ AB2 - Allgemeine Parabeln} 
\rhead{\includegraphics[scale=0.5]{/home/pfranz/hbo/logo.png}} 
\lfoot{} 
\cfoot{} 
\rfoot{} 
\usepackage{amsmath}  
\usepackage{cancel} 
\usepackage{pgf,tikz} 
\usetikzlibrary{arrows} 
\newtheoremstyle{note} 
{16pt}  % Platz zwischen Kopf und voherigem Text 
{16pt}  % und nachfolgendem Text 
{}     % Schriftart des Koerpers 
{}     % mit \parindent Einzug 
{\bf}  % Schriftart des Kopfes 
{:}     % Nach Bedarf z.B. Doppelpunkt nach dem Kopf 
{0.5em} % Platz zwischen Kopf und Koerper 
{}     % Kopfname 

\theoremstyle{note} 
\newtheorem{aufgabe}{Aufgabe} 



\begin{document} 
    \begin{flushleft}
Name: \hspace{12cm} Datum:\begin{aufgabe} ~ \\ 
Erstelle f\"ur die Funktion $f(x)=- 0.5 \left(x + 1\right)^{2} + 2.5$ eine Wertetabelle und zeichne den dazugeh\"origen Graphen im Bereich von $x=-3$ bis $x=3$ in ein Koordinatensystem. \\ 
\renewcommand{\arraystretch}{1.0} 
\begin{tabular}{c|c|c|c|c|c|c}
-3 & -2 & -1 & 0 & 1 & 2 & 3\\ \hline 
0.5 & 2.0 & 2.5 & 2.0 & 0.5 & -2.0 & -5.5\\ 

\end{tabular} 

\definecolor{cqcqcq}{rgb}{0.75,0.75,0.75} 
\begin{tikzpicture}[domain=-3:3,scale=1][line cap=round,line join=round,>=triangle 45,x=1.0cm,y=1.0cm]\draw[color=cqcqcq,dash pattern=on 2pt off 2pt, xstep=1.0cm,ystep=1.0cm] (-3.0,-3.0) grid (3.0,3.0); 
\draw[->] (-3.0,0) -- (3.0,0) node[below] {\footnotesize $x$}; 
\foreach \x in {-3, -2, -1,  1, 2}
\draw[shift={(\x,0)},color=black] (0pt,2pt) -- (0pt,-2pt) node[below] {\footnotesize $\x$};\draw[->,color=black] (0,-3.0) -- (0,3.0) node[right] {\footnotesize $y$}; 
\foreach \y in {-3, -2, -1,  1, 2}
\draw[shift={(0,\y)},color=black] (2pt,0pt) -- (-2pt,0pt) node[left] {\footnotesize $\y$}; 
\draw[color=black] (0pt,-10pt) node[right] {\footnotesize $0$}; 
\begin{scope} 
\clip (-3,-3) rectangle (3,3); 
\draw[color=black] plot[smooth] function{-0.5*(x + 1)**2 + 2.5}; 
\draw (-0.5,2.7) node[right] {$\scriptscriptstyle f$}; 
\end{scope} 
\end{tikzpicture}\end{aufgabe} 
\begin{aufgabe} ~ \\ 
Gib f\"ur die folgenden Parabeln Scheitelpunkt, Symmetrieachse, Nullstellen und \"Offnung an. Gib weiter an, ob die Parabeln durch Streckung oder Stauchung aus der Normalparabel entstehen. 
\begin{minipage}{0.15\textwidth} 
\definecolor{cqcqcq}{rgb}{0.75,0.75,0.75} 
\begin{tikzpicture}[domain=-3:3,scale=0.5][line cap=round,line join=round,>=triangle 45,x=1.0cm,y=1.0cm]\draw[color=cqcqcq,dash pattern=on 2pt off 2pt, xstep=1.0cm,ystep=1.0cm] (-3.0,-3.0) grid (3.0,3.0); 
\draw[->] (-3.0,0) -- (3.0,0) node[below] {\footnotesize $x$}; 
\foreach \x in {-3, -2, -1,  1, 2}
\draw[shift={(\x,0)},color=black] (0pt,2pt) -- (0pt,-2pt) node[below] {\footnotesize $\x$};\draw[->,color=black] (0,-3.0) -- (0,3.0) node[right] {\footnotesize $y$}; 
\foreach \y in {-3, -2, -1,  1, 2}
\draw[shift={(0,\y)},color=black] (2pt,0pt) -- (-2pt,0pt) node[left] {\footnotesize $\y$}; 
\draw[color=black] (0pt,-10pt) node[right] {\footnotesize $0$}; 
\begin{scope} 
\clip (-3,-3) rectangle (3,3); 
\draw[color=black] plot[smooth] function{(x - 1)**2 - 2}; 
\draw (1.5,-1.8) node[right] {$\scriptscriptstyle f_1$}; 
\draw[color=black] plot[smooth] function{-(x - 1)**2 - 2}; 
\draw (1.5,-1.8) node[right] {$\scriptscriptstyle f_2$}; 
\draw[color=black] plot[smooth] function{-3*(x + 2)**2/4 + 2}; 
\draw (-1.5,2.2) node[right] {$\scriptscriptstyle f_3$}; 
\end{scope} 
\end{tikzpicture}\end{minipage} 
\begin{minipage}{0.1\textwidth} 
 ~ \end{minipage} 
\begin{minipage}{0.65\textwidth} 
\renewcommand{\arraystretch}{1.3} 
\begin{tabular}{c|c|c|c|c|c}
 & Scheitelpunkt & Symmetrieachse & Nullstellen & \"Offnung & Form\\ \hline 
$f_1$ & $S(1|-2)$ & $x=1$ & $x_1\approx 2,\hspace{-2pt}41\quad x_2\approx-0,\hspace{-2pt}41$ & oben & normal\\ \hline 
$f_2$ & $S(1|-2)$ & $x=1$ & Keine Nullstelle & unten & normal\\ \hline 
$f_3$ & $S(-2|2)$ & $x=-2$ & $x_1\approx -3,\hspace{-2pt}63\quad x_2\approx-0,\hspace{-2pt}37$ & unten & gestaucht\\ 

\end{tabular} 

\end{minipage} 

\end{aufgabe} 
\begin{aufgabe} ~ \\ 
Streiche die Graphen, die nicht zur Funktionsgleichung passen. \\ 
\begin{minipage}{0.25\textwidth} 
\vspace{1em} 
$f_1(x)=\frac{2}{3} \left(x - 2\right)^{2} - 1$\\[1em] 
\end{minipage} 
\begin{minipage}{0.1\textwidth} 
 ~ \end{minipage} 
\begin{minipage}{0.55\textwidth} 
\definecolor{cqcqcq}{rgb}{0.75,0.75,0.75} 
\begin{tikzpicture}[domain=-3:3,scale=0.4][line cap=round,line join=round,>=triangle 45,x=1.0cm,y=1.0cm]\draw[color=cqcqcq,dash pattern=on 2pt off 2pt, xstep=1.0cm,ystep=1.0cm] (-3.0,-3.0) grid (3.0,3.0); 
\draw[->] (-3.0,0) -- (3.0,0) node[below] {\footnotesize $x$}; 
\foreach \x in {-3, -2, -1,  1, 2}
\draw[shift={(\x,0)},color=black] (0pt,2pt) -- (0pt,-2pt) node[below] {\footnotesize $\x$};\draw[->,color=black] (0,-3.0) -- (0,3.0) node[right] {\footnotesize $y$}; 
\foreach \y in {-3, -2, -1,  1, 2}
\draw[shift={(0,\y)},color=black] (2pt,0pt) -- (-2pt,0pt) node[left] {\footnotesize $\y$}; 
\draw[color=black] (0pt,-10pt) node[right] {\footnotesize $0$}; 
\begin{scope} 
\clip (-3,-3) rectangle (3,3); 
\draw[color=black] plot[smooth] function{-4*(x + 2)**2 - 1}; 
\draw (-3,-3) -- (3,3); 
\end{scope} 
\end{tikzpicture}\definecolor{cqcqcq}{rgb}{0.75,0.75,0.75} 
\begin{tikzpicture}[domain=-3:3,scale=0.4][line cap=round,line join=round,>=triangle 45,x=1.0cm,y=1.0cm]\draw[color=cqcqcq,dash pattern=on 2pt off 2pt, xstep=1.0cm,ystep=1.0cm] (-3.0,-3.0) grid (3.0,3.0); 
\draw[->] (-3.0,0) -- (3.0,0) node[below] {\footnotesize $x$}; 
\foreach \x in {-3, -2, -1,  1, 2}
\draw[shift={(\x,0)},color=black] (0pt,2pt) -- (0pt,-2pt) node[below] {\footnotesize $\x$};\draw[->,color=black] (0,-3.0) -- (0,3.0) node[right] {\footnotesize $y$}; 
\foreach \y in {-3, -2, -1,  1, 2}
\draw[shift={(0,\y)},color=black] (2pt,0pt) -- (-2pt,0pt) node[left] {\footnotesize $\y$}; 
\draw[color=black] (0pt,-10pt) node[right] {\footnotesize $0$}; 
\begin{scope} 
\clip (-3,-3) rectangle (3,3); 
\draw[color=black] plot[smooth] function{2*(x - 2)**2/3 - 1}; 
\end{scope} 
\end{tikzpicture}\definecolor{cqcqcq}{rgb}{0.75,0.75,0.75} 
\begin{tikzpicture}[domain=-3:3,scale=0.4][line cap=round,line join=round,>=triangle 45,x=1.0cm,y=1.0cm]\draw[color=cqcqcq,dash pattern=on 2pt off 2pt, xstep=1.0cm,ystep=1.0cm] (-3.0,-3.0) grid (3.0,3.0); 
\draw[->] (-3.0,0) -- (3.0,0) node[below] {\footnotesize $x$}; 
\foreach \x in {-3, -2, -1,  1, 2}
\draw[shift={(\x,0)},color=black] (0pt,2pt) -- (0pt,-2pt) node[below] {\footnotesize $\x$};\draw[->,color=black] (0,-3.0) -- (0,3.0) node[right] {\footnotesize $y$}; 
\foreach \y in {-3, -2, -1,  1, 2}
\draw[shift={(0,\y)},color=black] (2pt,0pt) -- (-2pt,0pt) node[left] {\footnotesize $\y$}; 
\draw[color=black] (0pt,-10pt) node[right] {\footnotesize $0$}; 
\begin{scope} 
\clip (-3,-3) rectangle (3,3); 
\draw[color=black] plot[smooth] function{(x + 1)**2/4 - 1}; 
\draw (-3,-3) -- (3,3); 
\end{scope} 
\end{tikzpicture}\end{minipage} 

\end{aufgabe} 
\begin{aufgabe} ~ \\ 
Unterstreiche die Funktionsgleichung, die zum Graphen passt. \\ 
\begin{minipage}{0.2\textwidth} 
\definecolor{cqcqcq}{rgb}{0.75,0.75,0.75} 
\begin{tikzpicture}[domain=-3:3,scale=0.6][line cap=round,line join=round,>=triangle 45,x=1.0cm,y=1.0cm]\draw[color=cqcqcq,dash pattern=on 2pt off 2pt, xstep=1.0cm,ystep=1.0cm] (-3.0,-3.0) grid (3.0,3.0); 
\draw[->] (-3.0,0) -- (3.0,0) node[below] {\footnotesize $x$}; 
\foreach \x in {-3, -2, -1,  1, 2}
\draw[shift={(\x,0)},color=black] (0pt,2pt) -- (0pt,-2pt) node[below] {\footnotesize $\x$};\draw[->,color=black] (0,-3.0) -- (0,3.0) node[right] {\footnotesize $y$}; 
\foreach \y in {-3, -2, -1,  1, 2}
\draw[shift={(0,\y)},color=black] (2pt,0pt) -- (-2pt,0pt) node[left] {\footnotesize $\y$}; 
\draw[color=black] (0pt,-10pt) node[right] {\footnotesize $0$}; 
\begin{scope} 
\clip (-3,-3) rectangle (3,3); 
\draw[color=black] plot[smooth] function{-(x - 2)**2}; 
\end{scope} 
\end{tikzpicture}\end{minipage} 
\begin{minipage}{0.1\textwidth} 
 ~ \end{minipage} 
\begin{minipage}{0.6\textwidth} 
\vspace{1em} 
$\cancel{f_1(x)=\frac{1}{2} \left(x + 2\right)^{2} - 2}$\\[1em] 
$\cancel{f_2(x)=- \frac{1}{2} \left(x - 1\right)^{2}}$\\[1em] 
$f_3(x)=- \left(x - 2\right)^{2}$\\[1em] 
$\cancel{f_4(x)=- 3 \left(x - 2\right)^{2} + 1}$\\[1em] 
\end{minipage} 

\end{aufgabe} 
\begin{aufgabe} ~ \\ 
Bringe die Funktionsgleichungen auf Normalform. \\ 
\begin{multicols}{3} 
\begin{enumerate}[a)] 
\item 
$f_1(x)=x^{2} + 1$ \\ 
$f_1(x)=x^{2} + 1$ \\ 

\item 
$f_2(x)=x^{2} - 2$ \\ 
$f_2(x)=x^{2} - 2$ \\ 

\item 
$f_3(x)=2 \left(x + 1\right)^{2} + 2$ \\ 
$f_3(x)=2 x^{2} + 4 x + 4$ \\ 

\item 
$f_4(x)=x^{2} + 2$ \\ 
$f_4(x)=x^{2} + 2$ \\ 

\item 
$f_5(x)=3 x^{2} + 2$ \\ 
$f_5(x)=3 x^{2} + 2$ \\ 

\item 
$f_6(x)=3 x^{2} + 2$ \\ 
$f_6(x)=3 x^{2} + 2$ \\ 

\end{enumerate} 
\end{multicols} 
\end{aufgabe} 
\begin{aufgabe} ~ \\ 
Bringe die Funktionsgleichungen auf Scheitelpunktform. \\ 
\begin{multicols}{3} 
\begin{enumerate}[a)] 
\item 
$f_1(x)=- 2 x^{2} - 8 x - 8$ \\ 
$f_1(x)=- 2 \left(x + 2\right)^{2}$ \\ 

\item 
$f_2(x)=x^{2} + 6 x + 8$ \\ 
$f_2(x)=\left(x + 3\right)^{2} - 1$ \\ 

\item 
$f_3(x)=- 2 x^{2} - 12 x - 19$ \\ 
$f_3(x)=- 2 \left(x + 3\right)^{2} - 1$ \\ 

\item 
$f_4(x)=- x^{2} - 2 x + 1$ \\ 
$f_4(x)=- \left(x + 1\right)^{2} + 2$ \\ 

\item 
$f_5(x)=x^{2} - 4 x + 5$ \\ 
$f_5(x)=\left(x - 2\right)^{2} + 1$ \\ 

\item 
$f_6(x)=- 3 x^{2} - 6 x - 3$ \\ 
$f_6(x)=- 3 \left(x + 1\right)^{2}$ \\ 

\end{enumerate} 
\end{multicols} 
\end{aufgabe} 
\end{flushleft} 
    \end{document}