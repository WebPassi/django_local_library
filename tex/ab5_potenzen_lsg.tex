\documentclass[12pt,fleqn]{article}
\usepackage[utf8]{inputenc}
\usepackage{paralist} 
\usepackage{amssymb}  
\usepackage{amsthm}   
\usepackage{eurosym} 
\usepackage{multicol} 
\usepackage{framed}
\usepackage[left=1.5cm,right=1.5cm,top=1.5cm,bottom=0.5cm]{geometry} 
\usepackage{fancyheadings} 
\pagestyle{fancy} 
\headheight1.6cm 
\lhead{Name:} 
\chead{\hspace{5cm} Klasse: \qquad\qquad Datum: \qquad\qquad} 
\rhead{\includegraphics[scale=0.3]{/home/pfranz/hbo/logo.png}} 
\lfoot{} 
\cfoot{} 
\rfoot{} 
\usepackage{amsmath}  
\usepackage{cancel} 
\usepackage{pgf,tikz} 
\usetikzlibrary{arrows} 
\newtheoremstyle{aufg} 
{16pt}  % Platz zwischen Kopf und voherigem Text 
{16pt}  % und nachfolgendem Text 
{}     % Schriftart des Koerpers 
{}     % mit \parindent Einzug 
{\bf}  % Schriftart des Kopfes 
{:}     % Nach Bedarf z.B. Doppelpunkt nach dem Kopf 
{0.5em} % Platz zwischen Kopf und Koerper 
{}     % Kopfname 

\theoremstyle{aufg} 
\newtheorem{aufgabe}{Aufgabe} 



\newtheoremstyle{bsp} 
{16pt}  % Platz zwischen Kopf und voherigem Text 
{16pt}  % und nachfolgendem Text 
{}     % Schriftart des Koerpers 
{}     % mit \parindent Einzug 
{\em}  % Schriftart des Kopfes 
{:}     % Nach Bedarf z.B. Doppelpunkt nach dem Kopf 
{0.5em} % Platz zwischen Kopf und Koerper 
{}     % Kopfname 

\theoremstyle{bsp} 
\newtheorem{beispiel}{Beispiel} 


\begin{document} 
    \begin{flushleft}
\begin{center}AB 5 - Potenzen\end{center}\begin{aufgabe} ~ \\ 
Schreibe ohne Potenz als Multiplikationsaufgabe. \\ 
\begin{multicols}{3} 
\begin{enumerate}[a)] 
\item 
$5^{7}=$$5\cdot5\cdot5\cdot5\cdot5\cdot5\cdot5$
\item 
$6^{2}=$$6\cdot6$
\item 
$3^{5}=$$3\cdot3\cdot3\cdot3\cdot3$
\end{enumerate} 
\end{multicols} 
\end{aufgabe} 
\begin{aufgabe} ~ \\ 
Schreibe als Potenz. \\ 
\begin{multicols}{3} 
\begin{enumerate}[a)] 
\item 
$2\cdot2\cdot2\cdot2=$$2^{4}$
\item 
$7\cdot7=$$7^{2}$
\item 
$5\cdot5\cdot5\cdot5=$$5^{4}$
\end{enumerate} 
\end{multicols} 
\end{aufgabe} 
\begin{aufgabe} ~ \\ 
Schreibe als Bruch. \\ 
\begin{multicols}{3} 
\begin{enumerate}[a)] 
\item 
$6^{-3}=$$\frac{1}{6^{3}}$
\item 
$4^{-8}=$$\frac{1}{4^{8}}$
\item 
$9^{-4}=$$\frac{1}{9^{4}}$
\end{enumerate} 
\end{multicols} 
\end{aufgabe} 
\begin{aufgabe} ~ \\ 
Schreibe als Wurzel. \\ 
\begin{multicols}{3} 
\begin{enumerate}[a)] 
\item 
$5^{\frac{8}{11}}=$$\sqrt[11]{5}^{\,8}$
\item 
$4^{\frac{7}{3}}=$$\sqrt[3]{4}^{\,7}$
\item 
$3^{\frac{8}{7}}=$$\sqrt[7]{3}^{\,8}$
\end{enumerate} 
\end{multicols} 
\end{aufgabe} 
\begin{aufgabe} ~ \\ 
Schreibe als Potenz. \\ 
\begin{multicols}{3} 
\begin{enumerate}[a)] 
\item 
$\sqrt[14]{7}^{\,11}=$$7^{\frac{11}{14}}$
\item 
$\sqrt[2]{2}^{\,7}=$$2^{\frac{7}{2}}$
\item 
$\sqrt[5]{5}^{\,14}=$$5^{\frac{14}{5}}$
\end{enumerate} 
\end{multicols} 
\end{aufgabe} 
\begin{aufgabe} ~ \\ 
Schreibe als eine Potenz. \\ 
\begin{multicols}{3} 
\begin{enumerate}[a)] 
\item 
$3^{-5}\cdot3^{0}=$$3^{-5}$
\item 
$9^{8}\cdot9^{2}=$$9^{10}$
\item 
$6^{-4}\cdot6^{-4}=$$6^{-8}$
\end{enumerate} 
\end{multicols} 
\end{aufgabe} 
\begin{aufgabe} ~ \\ 
Schreibe als eine Potenz. \\ 
\begin{multicols}{3} 
\begin{enumerate}[a)] 
\item 
$\frac{2^{-10}}{2^{-8}}=$$2^{-2}$
\item 
$\frac{2^{-5}}{2^{-5}}=$$2^{0}$
\item 
$\frac{6^{7}}{6^{4}}=$$6^{3}$
\end{enumerate} 
\end{multicols} 
\end{aufgabe} 
\begin{aufgabe} ~ \\ 
Berechne mit dem Taschenrechner. \\ 
\begin{multicols}{2} 
\begin{enumerate}[a)] 
\item 
$1\cdot5^{5}+6\cdot8^{-2}=$$3125.09375$
\item 
$-5\cdot3^{-1}+4\cdot4^{2}=$$62.3333333333$
\item 
$-4\cdot7^{5}+10\cdot6^{-2}=$$-67227.7222222$
\item 
$-6\cdot6^{-2}+1\cdot8^{-1}=$$-0.0416666666667$
\end{enumerate} 
\end{multicols} 
\end{aufgabe} 
\end{flushleft} 
\end{document}