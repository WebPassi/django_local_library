\documentclass[12pt,fleqn]{article}
\usepackage[utf8]{inputenc}
\usepackage{paralist} 
\usepackage{amssymb}  
\usepackage{amsthm}   
\usepackage{eurosym} 
\usepackage{multicol} 
\usepackage[left=1.5cm,right=1.5cm,top=1.5cm,bottom=0.5cm]{geometry} 
\usepackage{fancyheadings} 
\pagestyle{fancy} 
\headheight1.6cm 
\lhead{\today} 
\chead{WA Pauli} 
\rhead{\includegraphics[scale=0.5]{logo.png}} 
\lfoot{} 
    \cfoot{} 
    \rfoot{} 
\usepackage{amsmath}  
\usepackage{cancel} 
\usepackage{pgf,tikz} 
\usetikzlibrary{arrows} 
\newtheoremstyle{aufg} 
{16pt}  % Platz zwischen Kopf und voherigem Text 
{16pt}  % und nachfolgendem Text 
{}     % Schriftart des Koerpers 
{}     % mit \parindent Einzug 
{\bf}  % Schriftart des Kopfes 
{:}     % Nach Bedarf z.B. Doppelpunkt nach dem Kopf 
{0.5em} % Platz zwischen Kopf und Koerper 
{}     % Kopfname 

\theoremstyle{aufg} 
\newtheorem{aufgabe}{Aufgabe} 



\newtheoremstyle{bsp} 
{16pt}  % Platz zwischen Kopf und voherigem Text 
{16pt}  % und nachfolgendem Text 
{}     % Schriftart des Koerpers 
{}     % mit \parindent Einzug 
{\em}  % Schriftart des Kopfes 
{:}     % Nach Bedarf z.B. Doppelpunkt nach dem Kopf 
{0.5em} % Platz zwischen Kopf und Koerper 
{}     % Kopfname 

\theoremstyle{bsp} 
\newtheorem{beispiel}{Beispiel} 



\begin{document} 
    \begin{flushleft}
\renewcommand{\arraystretch}{2.15} 
\begin{tabular}{|p{10cm}|p{2cm}|p{2cm}|p{2cm}|} 
\hline 
\hspace{2cm} Punkte von \qquad30\qquad Punkten erreicht & \hspace{1.2cm} NP & G & E \\ 
\hline 
\end{tabular} \\[1em]    
\begin{center}{\Large Schriftliche Arbeit}\end{center} 
{\bf Zeit: }45 Minuten\\ 
\begin{aufgabe}\framebox{\qquad/3} ~ \\ 
\begin{multicols}{3} 
\begin{enumerate}[a)] 
\item 
$13y-12y=y$
\item 
$3b-13b=- 10 b$
\item 
$14b+11b=25 b$
\end{enumerate} 
\end{multicols} 
\end{aufgabe} 
\begin{aufgabe}\framebox{\qquad/4} ~ \\ 
\begin{multicols}{2} 
\begin{enumerate}[a)] 
\item 
$15b-16b-18b+14b=- 5 b$
\item 
$1x-19b+20b-6x=b - 5 x$
\end{enumerate} 
\end{multicols} 
\end{aufgabe} 
\begin{aufgabe}\framebox{\qquad/5} ~ \\ 
\begin{multicols}{2} 
\begin{enumerate}[a)] 
\item 
$4(7a-1)=28 a - 4$
\item 
$(11a+10)(19a+8)=209 a^{2} + 278 a + 80$
\end{enumerate} 
\end{multicols} 
\end{aufgabe} 
\begin{aufgabe}\framebox{\qquad/6} ~ \\ 
\begin{multicols}{3} 
\begin{enumerate}[a)] 
\item 
$- x = -1$\\$x=1$
\item 
$x + 2 = 2$\\$x=0$
\item 
$- 3 x - 3 = 3 x - 1$\\$x=- \frac{1}{3}$
\end{enumerate} 
\end{multicols} 
\end{aufgabe} 
\begin{aufgabe}\framebox{\qquad/4} ~ \\ 
Herr Stein leiht sich von der Bank $\mathrm{\,6000}$ Euro zu einem Zinssatz von $\mathrm{\,6}$ \% pro Jahr. Berechne die Zinsen, die er nach einem Jahr an die Bank bezahlen muss.
$360.0$\end{aufgabe} 
\begin{aufgabe}\framebox{\qquad/4} ~ \\ 
Ihr Chef hat ihnen eine $5\%$-ige Gehaltserh\"ohung versprochen. Im n\"achsten Monat bekommst du statt $\mathrm{\,3398.91}$ Euro nun $\mathrm{\,3596.72}$ Euro. \"Uberpr\"ufe, ob der Chef Ihnen die versprochene Lohnerh\"ohung gezahlt hat. 
$0.0581980693811$\end{aufgabe} 
\begin{aufgabe}\framebox{\qquad/4} ~ \\ 
Die Arbeitslosenquote in Berlin ist laut einer Studie um $\mathrm{\,2}$ \% gestiegen. Das entspricht $\mathrm{\,298000}$ mehr Arbeitsuchende als vor einem Jahr. Bestimme, wieviele Arbeitslose Berlin jetzt und vor einem Jahr hatte.  
$14900000.0,15198000.0$\end{aufgabe} 
\end{flushleft} 
    \end{document}