\documentclass[12pt,fleqn]{article}
\usepackage[utf8]{inputenc}
\usepackage{paralist} 
\usepackage{amssymb}  
\usepackage{amsthm}   
\usepackage{eurosym} 
\usepackage{multicol} 
\usepackage[left=1.5cm,right=1.5cm,top=1.5cm,bottom=0.5cm]{geometry} 
\usepackage{fancyheadings} 
\pagestyle{fancy} 
\headheight1.6cm 
\lhead{MA-1} 
\chead{Lineare Funktionen} 
\rhead{\includegraphics[scale=0.5]{logo.png}} 
\lfoot{} 
\cfoot{} 
\rfoot{} 
\usepackage{amsmath}  
\usepackage{cancel} 
\usepackage{pgf,tikz} 
\usetikzlibrary{arrows} 
\newtheoremstyle{aufg} 
{16pt}  % Platz zwischen Kopf und voherigem Text 
{16pt}  % und nachfolgendem Text 
{}     % Schriftart des Koerpers 
{}     % mit \parindent Einzug 
{\bf}  % Schriftart des Kopfes 
{:}     % Nach Bedarf z.B. Doppelpunkt nach dem Kopf 
{0.5em} % Platz zwischen Kopf und Koerper 
{}     % Kopfname 

\theoremstyle{aufg} 
\newtheorem{aufgabe}{Aufgabe} 



\newtheoremstyle{bsp} 
{16pt}  % Platz zwischen Kopf und voherigem Text 
{16pt}  % und nachfolgendem Text 
{}     % Schriftart des Koerpers 
{}     % mit \parindent Einzug 
{\em}  % Schriftart des Kopfes 
{:}     % Nach Bedarf z.B. Doppelpunkt nach dem Kopf 
{0.5em} % Platz zwischen Kopf und Koerper 
{}     % Kopfname 

\theoremstyle{bsp} 
\newtheorem{beispiel}{Beispiel} 



\begin{document} 
    \begin{flushleft}
\begin{center}Lineare Funktionen\end{center} 
 04.09.15 \\[2em]Gegeben ist die Funktion f mit\[f(x)=\frac{x}{4} + 3\; . \]\begin{enumerate}[a)] 
\item 
Bestimmen Sie die Nullstellen von $f$. \\ 
{\bf Nullstellen:} 
\begin{align*} 
f(x)&=0 \\ 
\frac{x}{4} + 3&=0 \\ 
\text{Mit CAS:} \\ 
x&=-12\end{align*} 
$\Rightarrow$ Nullstelle bei $N(-12|0)$ \\ 

\item 
Zeichnen Sie den Graphen von $f$. \\ 
\definecolor{cqcqcq}{rgb}{0.75,0.75,0.75} 
\begin{tikzpicture}[line cap=round,line join=round,>=triangle 45,x=1.0cm,y=1.0cm] 
\draw [color=cqcqcq,dash pattern=on 2pt off 2pt, xstep=1.0cm,ystep=1.0cm] (-3.33,-3.57) grid (3.29,3.53); 
\draw[->,color=black] (-3.33,0) -- (3.29,0); 
\foreach \x in {-3,-2,-1,1,2,3} 
\draw[shift={(\x,0)},color=black] (0pt,2pt) -- (0pt,-2pt) node[below] {\footnotesize $\x$}; 
\draw[color=black] (3,0.07) node [anchor=south west] { x}; 
\draw[->,color=black] (0,-3.57) -- (0,3.53); 
\foreach \y in {-3,-2,-1,1,2,3} 
\draw[shift={(0,\y)},color=black] (2pt,0pt) -- (-2pt,0pt) node[left] {\footnotesize $\y$}; 
\draw[color=black] (0.09,3.17) node [anchor=west] { y}; 
\draw[color=black] (0pt,-10pt) node[right] {\footnotesize $0$}; 
\clip(-3.33,-3.57) rectangle (3.29,3.53);\draw [domain=-3.33:3.29] plot(\x,{(1/4*\x+3}); 
\end{tikzpicture} 
\hspace{1cm}
\item 
Bestimmen Sie den Funktionswert an der Stelle $x=-4$. \\ 
{\bf Funktionswert:} \\ 
$f(-4)=2$\qquad (mit CAS)
\item 
Bestimmen Sie, an welcher Stelle die Funktion den Wert $y=5$ annimmt. \\ 
{\bf Funktionsstelle:} 
\begin{align*} 
f(x)&=5\\ 
\frac{x}{4} + 3&=0 \\ 
\text{Mit CAS:} \\ 
x&=8 
\end{align*} 
\clearpage
\item 
Untersuchen Sie die Steigung von $f$ sowohl qualitativ (fallend/steigend) als auch quantitativ. Geben Sie hierzu auch die Steigung in Prozent und den Steigungswinkel an. \\ 
{\bf Steigung} (in Prozent):\quad $m=\frac{1}{4}=25.0\%$ \\ 
{\bf Steigungswinkel:} \\ 
\begin{align*} 
\tan(\alpha) &=\frac{1}{4}\\ 
\text{mit CAS:} \\ 
\alpha &=14.0^{\circ} 
\end{align*} 

\item 
Gegeben ist eine weitere Funktion $g$, deren Graph durch die Punkte $A(2|0)$ und $B(-4|6)$ verl\"auft. Bestimmen Sie die Funktionsgleichung von $g$. \\ 
{\bf Geradengleichung aufstellen:} \quad $y=mx+c$ (*) \\ 
{\bf Steigung bestimmen:} \\ 
$m=\frac{(6)-(0)}{(-4)-(2)}=-1$ \\ 
$y$-Wert, $x$-Wert und $m$ in (*) einsetzen: 
\begin{align*} 
0&=-1\cdot2+ c \\ 
2&= c 
\end{align*} \\ 
{\bf Funktionsgleichung:} 
\[g(x)=- x + 2\]
\item 
Untersuchen Sie, ob sich $f$ und $g$ schneiden und bestimmen Sie gegebenenfalls den Schnittpunkt. \\ 
{\bf Schnittpunkt:} 
\begin{align*} 
f(x)&=g(x) \\ 
\frac{x}{4} + 3&=- x + 2 \\ 
\text{mit CAS:} \\ 
x&=- \frac{4}{5}\end{align*} 
$\Rightarrow$ Schnittpunkt bei $N(- \frac{4}{5}|\frac{14}{5})$ \\ 

\item 
Bestimmen Sie den Schnittwinkel zwischen $f$ und $g$. \\ 
{\bf Schnittwinkel:} \\ 
\begin{align*} 
\tan(\alpha) &=\left|\frac{m_1-m_2}{1+m_1\cdot m_2}\right| \\ 
\tan(\alpha) &=\left|\frac{(\frac{1}{4})-(-1)}{1+(\frac{1}{4})(-1)}\right| \\ 
\text{mit CAS:} \\ 
\alpha &=59.0^{\circ} 
\end{align*} 

\end{enumerate} 
\end{flushleft} 
\end{document}