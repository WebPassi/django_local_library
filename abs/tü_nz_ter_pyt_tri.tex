\documentclass[12pt,fleqn]{article}
\usepackage[utf8]{inputenc}
\usepackage{paralist} 
\usepackage{amssymb}  
\usepackage{amsthm}   
\usepackage{eurosym} 
\usepackage{multicol} 

\usepackage{tkz-euclide}
\usepackage{wasysym}
\usepackage {graphicx}
\usepackage[ngerman]{babel}
\usepackage{amsmath}  % \align-Umgebung
\usetkzobj{all}


\usepackage[left=1.5cm,right=1.5cm,top=1.5cm,bottom=0.5cm]{geometry} 
\usepackage{fancyheadings} 
\pagestyle{fancy} 
\headheight1.6cm 
\lhead{} 
\chead{} 
\rhead{\includegraphics[scale=0.5]{logo.png}} 
\lfoot{} 
\cfoot{} 
\rfoot{} 
\usepackage{amsmath}  
\usepackage{cancel} 
\usepackage{pgf,tikz} 
\usetikzlibrary{arrows} 
\newtheoremstyle{aufg} 
{16pt}  % Platz zwischen Kopf und voherigem Text 
{16pt}  % und nachfolgendem Text 
{}     % Schriftart des Koerpers 
{}     % mit \parindent Einzug 
{\bf}  % Schriftart des Kopfes 
{:}     % Nach Bedarf z.B. Doppelpunkt nach dem Kopf 
{0.5em} % Platz zwischen Kopf und Koerper 
{}     % Kopfname 

\theoremstyle{aufg} 
\newtheorem{aufgabe}{Aufgabe} 



\newtheoremstyle{bsp} 
{16pt}  % Platz zwischen Kopf und voherigem Text 
{16pt}  % und nachfolgendem Text 
{}     % Schriftart des Koerpers 
{}     % mit \parindent Einzug 
{\em}  % Schriftart des Kopfes 
{:}     % Nach Bedarf z.B. Doppelpunkt nach dem Kopf 
{0.5em} % Platz zwischen Kopf und Koerper 
{}     % Kopfname 

\theoremstyle{bsp} 
\newtheorem{beispiel}{Beispiel} 



\begin{document} 
    \begin{flushleft}
\begin{aufgabe} ~ \\ 
\begin{multicols}{3} 
\begin{enumerate}[a)] 
\item 
$2-10=$
\item 
$3-8=$
\item 
$5-4=$
\item 
$-5+6=$
\item 
$-8+8=$
\item 
$-8+10=$
\item 
$-10-1=$
\item 
$-8-1=$
\item 
$-4-4=$
\end{enumerate} 
\end{multicols} 
\end{aufgabe} 
\begin{aufgabe} ~ \\ 
\begin{multicols}{3} 
\begin{enumerate}[a)] 
\item 
$(-1)-1=$
\item 
$9+(+8)=$
\item 
$10+(-8)=$
\item 
$5-(-5)=$
\item 
$-(7)+(-9)=$
\item 
$-(-1)-(+10)=$
\end{enumerate} 
\end{multicols} 
\end{aufgabe} 
\begin{aufgabe} ~ \\ 
\begin{multicols}{3} 
\begin{enumerate}[a)] 
\item 
$5-7a-3a=$
\item 
$-7a+5-3a=$
\item 
$5-7a-3=$
\item 
$2-3a-8a=$
\item 
$-3a+2-8a=$
\item 
$2-3a-8=$
\end{enumerate} 
\end{multicols} 
\end{aufgabe} 
\begin{aufgabe} ~ \\ 
\begin{multicols}{3} 
\begin{enumerate}[a)] 
\item 
$2-(-10a)+(-2a)=$
\item 
$(-10a)-(+2)-(-2a)=$
\item 
$(-2)+(+10a)-(-2)=$
\item 
$2-(-10a)+(-2a)=$
\item 
$(-10a)-(+2)-(-2a)=$
\item 
$(-2)+(+10a)-(-2)=$
\end{enumerate} 
\end{multicols} 
\end{aufgabe} 

\begin{aufgabe}

Bestimme die fehlenden Gr\"o\ss{}en.

\begin  {minipage}[t]{9cm}
\begin{tikzpicture}[rotate=180]
\coordinate (O) at (0,0);
\coordinate (A) at (4,0);
\coordinate (B) at (0,2);
\draw (O)--(A) node[midway,sloped, above]{a};
\draw (A)--(B) node[rotate=0,midway,sloped, below]{7cm};
\draw (B)--(O) node[rotate=180,midway,sloped, below]{3cm};
\tkzMarkAngle[ size=0.65cm](A,O,B)
\tkzLabelAngle[pos = 0.35](A,O,B){$\cdot$}
\tkzMarkAngle[ size=1.4cm](B,A,O)
\tkzLabelAngle[pos = 1](B,A,O){$\alpha$}
\tkzMarkAngle[ size=0.65cm](O,B,A)
\tkzLabelAngle[pos = 0.35](O,B,A){$\beta$}
\end{tikzpicture}
\end{minipage}
\begin  {minipage}[t]{5cm}
\begin{tikzpicture}[rotate=30]
\coordinate (O) at (0,0);
\coordinate (A) at (4,0);
\coordinate (B) at (0,2);
\draw (O)--(A) node[rotate=30,midway,sloped, below]{y};
\draw (A)--(B) node[rotate=30,midway,sloped, above]{x};
\draw (B)--(O) node[rotate=30,midway,sloped, below]{5cm};
\tkzMarkAngle[ size=0.65cm](A,O,B)
\tkzLabelAngle[pos = 0.35](A,O,B){$\cdot$}
\tkzMarkAngle[ size=0.65cm](O,B,A)
\tkzLabelAngle[pos = 0.35](O,B,A){$\beta$}
\tkzMarkAngle[ size=1.4cm](B,A,O)
\tkzLabelAngle[pos = 1](B,A,O){$33^\circ$}
\end{tikzpicture}
\end{minipage}

\begin  {minipage}[t]{9cm}
a) \framebox{\qquad/6} 
\end{minipage}
\begin  {minipage}[t]{5cm}
b) \framebox{\qquad/7} 
\end{minipage}


\vspace{1cm}
%Dreieck 5> Tan  berechnen

\begin  {minipage}[t]{9cm}
\begin{tikzpicture}[rotate=220]
\coordinate (O) at (0,0);
\coordinate (A) at (4,0);
\coordinate (B) at (0,2);
\draw (O)--(A) node[rotate=40,midway,sloped, above]{3cm};
\draw (A)--(B) node[rotate=220,midway,sloped, above]{x};
\draw (B)--(O) node[rotate=220,midway,sloped, above]{5cm};

\tkzMarkAngle[ size=0.65cm](A,O,B)
\tkzLabelAngle[pos = 0.35](A,O,B){$\cdot$}

\tkzMarkAngle[ size=1.4cm](B,A,O)
\tkzLabelAngle[pos = 1.1](B,A,O){$60^\circ$}

\tkzMarkAngle[ size=0.8cm](O,B,A)
\tkzLabelAngle[pos = 0.5](O,B,A){$\beta$}

\end{tikzpicture}
\end{minipage}
%Dreieck 6> beliebig  berechnen
\begin  {minipage}[t]{5cm}
\begin{tikzpicture}[rotate=30]
\coordinate (O) at (0,0);
\coordinate (A) at (4,0);
\coordinate (B) at (0,2);
\draw (O)--(A) node[rotate=30,midway,sloped, below]{3cm};
\draw (A)--(B) node[rotate=30,midway,sloped, above]{5cm};
\draw (B)--(O) node[rotate=30,midway,sloped, below]{4cm};

\tkzMarkAngle[ size=0.65cm](A,O,B)
\tkzLabelAngle[pos = 0.35](A,O,B){$\cdot$}

\tkzMarkAngle[ size=0.8cm](O,B,A)
\tkzLabelAngle[pos = 0.5](O,B,A){$\alpha$}

\tkzMarkAngle[ size=1.4cm](B,A,O)
\tkzLabelAngle[pos = 1](B,A,O){$\beta$}


\end{tikzpicture}
\end{minipage}

\begin  {minipage}[t]{9cm}
c) \framebox{\qquad/4} \\
\end{minipage}
\begin  {minipage}[t]{5cm}
d) \framebox{\qquad/3} \\
\end{minipage}



\end{aufgabe}






\end{flushleft} 
\end{document}
