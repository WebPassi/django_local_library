\documentclass[12pt,fleqn]{article}
\usepackage[utf8]{inputenc}
\usepackage{paralist} 
\usepackage{amssymb}  
\usepackage{amsthm}   
\usepackage{eurosym} 
\usepackage{multicol} 
\usepackage[left=1.5cm,right=1.5cm,top=1.5cm,bottom=0.5cm]{geometry} 
\usepackage{fancyheadings} 
\pagestyle{fancy} 
\headheight1.6cm 
\lhead{} 
\chead{} 
\rhead{\includegraphics[scale=0.5]{logo.png}} 
\lfoot{} 
    \cfoot{} 
    \rfoot{} 
\usepackage{amsmath}  
\usepackage{cancel} 
\usepackage{pgf,tikz} 
\usetikzlibrary{arrows} 
\newtheoremstyle{aufg} 
{16pt}  % Platz zwischen Kopf und voherigem Text 
{16pt}  % und nachfolgendem Text 
{}     % Schriftart des Koerpers 
{}     % mit \parindent Einzug 
{\bf}  % Schriftart des Kopfes 
{:}     % Nach Bedarf z.B. Doppelpunkt nach dem Kopf 
{0.5em} % Platz zwischen Kopf und Koerper 
{}     % Kopfname 

\theoremstyle{aufg} 
\newtheorem{aufgabe}{Aufgabe} 



\newtheoremstyle{bsp} 
{16pt}  % Platz zwischen Kopf und voherigem Text 
{16pt}  % und nachfolgendem Text 
{}     % Schriftart des Koerpers 
{}     % mit \parindent Einzug 
{\em}  % Schriftart des Kopfes 
{:}     % Nach Bedarf z.B. Doppelpunkt nach dem Kopf 
{0.5em} % Platz zwischen Kopf und Koerper 
{}     % Kopfname 

\theoremstyle{bsp} 
\newtheorem{beispiel}{Beispiel} 



\begin{document} 
    \begin{flushleft}
\begin{aufgabe} ~ \\ 
\begin{multicols}{3} 
\begin{enumerate}[a)] 
\item 
$9+(+6)=$
\item 
$2+(-6)=$
\item 
$7-(1)=$
\item 
$5-(-4)=$
\item 
$-(9)+(-1)=$
\item 
$-(-6)-(+10)=$
\end{enumerate} 
\end{multicols} 
\end{aufgabe} 
\begin{aufgabe} ~ \\ 
\begin{multicols}{3} 
\begin{enumerate}[a)] 
\item 
$2(3+3)=$
\item 
$3(3-4)=$
\item 
$4(-3-2)=$
\item 
$(-3)(-3-4)=$
\item 
$(-3+1)(-1+3)=$
\item 
$(-4+4)(-4-3)=$
\end{enumerate} 
\end{multicols} 
\end{aufgabe} 
\begin{aufgabe} ~ \\ 
\begin{multicols}{2} 
\begin{enumerate}[a)] 
\item 
$4a+9a=$
\item 
$8c-4c=$
\item 
$-3x-12x=$
\item 
$-14t+3c-16t=$
\item 
$a^2+ab+ab+b^2=$
\item 
$a^2-ab-ab+b^2=$
\item 
$a^2-ab+ab-b^2=$
\item 
$9a^2+4a\cdot b-4a\cdot b+4b^2=$
\item 
$2a^2+1a\cdot b-1a\cdot b+1b^2=$
\item 
$7a^2+8a\cdot b-4a\cdot b+4b^2=$
\end{enumerate} 
\end{multicols} 
\end{aufgabe} 
\begin{aufgabe} ~ \\ 
\begin{multicols}{3} 
\begin{enumerate}[a)] 
\item 
$7(10a+8)=$
\item 
$5(1y-9x)=$
\item 
$-2(-4-6t)=$
\item 
$(4a+1)\cdot 10=$
\item 
$(5y-6x)\cdot 6=$
\item 
$(-3-1t)(-3)=$
\end{enumerate} 
\end{multicols} 
\end{aufgabe} 
\begin{aufgabe} ~ \\ 
\begin{multicols}{3} 
\begin{enumerate}[a)] 
\item 
$(3x+3)(8x+3)=$
\item 
$(10a-7)(4a-2)=$
\item 
$(-10x+7)(6x-9)=$
\item 
$(-3y+6)(-4y+3)=$
\item 
$-(2s-3)(-1t-7)=$
\item 
$(-7x-3)(9y+4)=$
\end{enumerate} 
\end{multicols} 
\end{aufgabe} 
\begin{aufgabe} ~ \\ 
\begin{multicols}{3} 
\begin{enumerate}[a)] 
\item 
$(a+b)^2=$
\item 
$(a-b)^2=$
\item 
$(a+b)(a-b)=$
\item 
$(5x+6y)^2=$
\item 
$(7r-7s)^2=$
\item 
$(3v+8t)(3v-8t)=$
\item 
$(4x+8y)^2=$
\item 
$(4r-9s)^2=$
\item 
$(9v+8t)(9v-8t)=$
\item 
$(7x+2y)^2=$
\item 
$(2r-3s)^2=$
\item 
$(5v+9t)(5v-9t)=$
\end{enumerate} 
\end{multicols} 
\end{aufgabe} 
\end{flushleft} 
\end{document}