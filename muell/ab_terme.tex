\documentclass[12pt,fleqn]{article}
\usepackage[utf8]{inputenc}
\usepackage{paralist} 
\usepackage{amssymb}  
\usepackage{amsthm}   
\usepackage{eurosym} 
\usepackage{multicol} 
\usepackage[left=1.5cm,right=1.5cm,top=1.5cm,bottom=0.5cm]{geometry} 
\usepackage{fancyheadings} 
\pagestyle{fancy} 
\headheight1.6cm 
\lhead{} 
\chead{} 
\rhead{\includegraphics[scale=0.5]{logo.png}} 
\lfoot{} 
    \cfoot{} 
    \rfoot{} 
\usepackage{amsmath}  
\usepackage{cancel} 
\usepackage{pgf,tikz} 
\usetikzlibrary{arrows} 
\newtheoremstyle{aufg} 
{16pt}  % Platz zwischen Kopf und voherigem Text 
{16pt}  % und nachfolgendem Text 
{}     % Schriftart des Koerpers 
{}     % mit \parindent Einzug 
{\bf}  % Schriftart des Kopfes 
{:}     % Nach Bedarf z.B. Doppelpunkt nach dem Kopf 
{0.5em} % Platz zwischen Kopf und Koerper 
{}     % Kopfname 

\theoremstyle{aufg} 
\newtheorem{aufgabe}{Aufgabe} 



\newtheoremstyle{bsp} 
{16pt}  % Platz zwischen Kopf und voherigem Text 
{16pt}  % und nachfolgendem Text 
{}     % Schriftart des Koerpers 
{}     % mit \parindent Einzug 
{\em}  % Schriftart des Kopfes 
{:}     % Nach Bedarf z.B. Doppelpunkt nach dem Kopf 
{0.5em} % Platz zwischen Kopf und Koerper 
{}     % Kopfname 

\theoremstyle{bsp} 
\newtheorem{beispiel}{Beispiel} 



\begin{document} 
    \begin{flushleft}
\begin{aufgabe} ~ \\ 
\begin{multicols}{3} 
\begin{enumerate}[a)] 
\item 
$4+(+1)=$
\item 
$10+(-6)=$
\item 
$4-(5)=$
\item 
$9-(-6)=$
\item 
$-(4)+(-10)=$
\item 
$-(-3)-(+8)=$
\end{enumerate} 
\end{multicols} 
\end{aufgabe} 
\begin{aufgabe} ~ \\ 
\begin{multicols}{3} 
\begin{enumerate}[a)] 
\item 
$3(2+4)=$
\item 
$2(4-1)=$
\item 
$3(-1-1)=$
\item 
$(-4)(-3-1)=$
\item 
$(-1+5)(-5+4)=$
\item 
$(-3+2)(-2-3)=$
\end{enumerate} 
\end{multicols} 
\end{aufgabe} 
\begin{aufgabe} ~ \\ 
\begin{multicols}{2} 
\begin{enumerate}[a)] 
\item 
$7a+8a=$
\item 
$13c-5c=$
\item 
$-19x-9x=$
\item 
$-16t+20c-13t=$
\item 
$a^2+ab+ab+b^2=$
\item 
$a^2-ab-ab+b^2=$
\item 
$a^2-ab+ab-b^2=$
\item 
$2a^2+8a\cdot b-1a\cdot b+1b^2=$
\item 
$3a^2+3a\cdot b-6a\cdot b+6b^2=$
\item 
$6a^2+6a\cdot b-6a\cdot b+6b^2=$
\end{enumerate} 
\end{multicols} 
\end{aufgabe} 
\begin{aufgabe} ~ \\ 
\begin{multicols}{3} 
\begin{enumerate}[a)] 
\item 
$9(10a+7)=$
\item 
$9(2y-5x)=$
\item 
$-1(-8-3t)=$
\item 
$(1a+4)\cdot 4=$
\item 
$(5y-6x)\cdot 4=$
\item 
$(-4-3t)(-10)=$
\end{enumerate} 
\end{multicols} 
\end{aufgabe} 
\begin{aufgabe} ~ \\ 
\begin{multicols}{3} 
\begin{enumerate}[a)] 
\item 
$(4x+6)(2x+9)=$
\item 
$(4a-5)(9a-3)=$
\item 
$(-10x+8)(1x-1)=$
\item 
$(-4y+1)(-9y+3)=$
\item 
$-(3s-1)(-4t-1)=$
\item 
$(-7x-10)(2y+3)=$
\end{enumerate} 
\end{multicols} 
\end{aufgabe} 
\begin{aufgabe} ~ \\ 
\begin{multicols}{3} 
\begin{enumerate}[a)] 
\item 
$(a+b)^2=$
\item 
$(a-b)^2=$
\item 
$(a+b)(a-b)=$
\item 
$(5x+5y)^2=$
\item 
$(4r-7s)^2=$
\item 
$(1v+3t)(1v-3t)=$
\item 
$(6x+8y)^2=$
\item 
$(1r-6s)^2=$
\item 
$(7v+3t)(7v-3t)=$
\item 
$(5x+5y)^2=$
\item 
$(3r-3s)^2=$
\item 
$(2v+3t)(2v-3t)=$
\end{enumerate} 
\end{multicols} 
\end{aufgabe} 
\end{flushleft} 
\end{document}