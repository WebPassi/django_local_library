\documentclass[12pt]{article}
\usepackage[utf8]{inputenc}
\usepackage{paralist} 
\usepackage{amssymb}  
\usepackage{amsthm}   
\usepackage{multicol} 
\usepackage[left=1.5cm,right=1.5cm,top=0cm,bottom=0cm]{geometry} 
\usepackage{amsmath}  
\usepackage{pgf,tikz} 
\usetikzlibrary{arrows} 
\pagestyle{empty} 
\newtheoremstyle{note} 
{16pt}  % Platz zwischen Kopf und voherigem Text 
{16pt}  % und nachfolgendem Text 
{}     % Schriftart des Koerpers 
{}     % mit \parindent Einzug 
{\bf}  % Schriftart des Kopfes 
{:}     % Nach Bedarf z.B. Doppelpunkt nach dem Kopf 
{0.5em} % Platz zwischen Kopf und Koerper 
{}     % Kopfname 

\theoremstyle{note} 
\newtheorem{aufgabe}{Aufgabe} 



\begin{document} 
    \begin{flushleft}
\begin{aufgabe} ~ \\ 
\begin{multicols}{3} 
\begin{enumerate}[a)] 
\item 
10 x = -4
\item 
3 x = 2
\item 
- 2 x = 10
\item 
- 10 x = 6
\item 
- 4 x = -3
\item 
- 10 x = 7
\item 
- 9 x = 4
\item 
- 5 x = 4
\item 
x = 8
\end{enumerate} 
\end{multicols} 
\end{aufgabe} 
\begin{aufgabe} ~ \\ 
\begin{multicols}{3} 
\begin{enumerate}[a)] 
\item 
10 x - 6 = 0
\item 
- 6 x - 5 = 0
\item 
6 x + 7 = 0
\item 
2 x - 5 = 0
\item 
- 9 x - 2 = 0
\item 
6 x - 4 = 0
\item 
- x - 5 = 0
\item 
7 x + 3 = 0
\item 
10 x + 6 = 0
\end{enumerate} 
\end{multicols} 
\end{aufgabe} 
\begin{aufgabe} ~ \\ 
\begin{multicols}{3} 
\begin{enumerate}[a)] 
\item 
- 5 x - 1 = -8
\item 
- 7 x + 3 = 0
\item 
8 x - 3 = 0
\item 
3 x + 5 = 9
\item 
- x = -7
\item 
6 x + 6 = -3
\item 
4 x - 4 = -10
\item 
- 3 x + 8 = -1
\item 
- 4 x + 4 = 6
\end{enumerate} 
\end{multicols} 
\end{aufgabe} 
\begin{aufgabe} ~ \\ 
\begin{multicols}{3} 
\begin{enumerate}[a)] 
\item 
- 3 x - 3 = 7 x + 7
\item 
x = - 4 x + 2
\item 
2 x + 7 = 7 x + 1
\item 
4 x + 2 = 9 x + 4
\item 
9 x + 2 = 3 x - 2
\item 
- x - 1 = - 6 x - 1
\item 
- 5 x + 5 = - 9 x + 1
\item 
- 3 x - 7 = - 6 x + 10
\item 
8 x - 2 = - 4 x + 2
\end{enumerate} 
\end{multicols} 
\end{aufgabe} 
\begin{aufgabe} ~ \\ 
\begin{multicols}{3} 
\begin{enumerate}[a)] 
\item 
10(8 x + 2) = -6
\item 
3(- x - 6) = -2
\item 
5(3 x - 7) = 6
\item 
- 5 x - 1 = 0
\item 
8(x + 8) = 8
\item 
7(5 x - 8) = -1
\item 
2(- 3 x + 4) = -1
\item 
- 4 x + 8 = -6
\item 
8(x + 9) = -3
\end{enumerate} 
\end{multicols} 
\end{aufgabe} 
\begin{aufgabe} ~ \\ 
\begin{multicols}{2} 
\begin{enumerate}[a)] 
\item 
-4(x) = -6(4 x - 7)
\item 
-10(9 x) = 6(10 x - 1)
\item 
-1(x + 5) = -8(- 8 x - 2)
\item 
-3(2 x - 2) = -9(- 6 x - 3)
\item 
-7(3 x - 7) = -6(7 x + 5)
\item 
-6(2 x + 3) = -5(- 5 x + 6)
\item 
6(x + 7) = -8(3 x)
\item 
3(- 5 x + 6) = - 3 x - 10
\item 
-4(- 6 x - 7) = 2(9 x - 5)
\end{enumerate} 
\end{multicols} 
\end{aufgabe} 
$ \scriptstyle- \frac{75}{8}$ , $ \scriptstyle- \frac{3}{7}$ , $ \scriptstyle- \frac{1}{5}$ , $ \scriptstyle\frac{7}{2}$ , $ \scriptstyle- \frac{2}{5}$ , $ \scriptstyle- \frac{3}{5}$ , $ \scriptstyle\frac{3}{5}$ , $ \scriptstyle- \frac{1}{2}$ , $ \scriptstyle\frac{7}{3}$ , $ \scriptstyle- \frac{19}{3}$ , $ \scriptstyle3$ , $ \scriptstyle- \frac{16}{3}$ , $ \scriptstyle\frac{2}{3}$ , $ \scriptstyle\frac{2}{5}$ , $ \scriptstyle- \frac{7}{5}$ , $ \scriptstyle- \frac{2}{3}$ , $ \scriptstyle- \frac{5}{6}$ , $ \scriptstyle\frac{7}{5}$ , $ \scriptstyle7$ , $ \scriptstyle\frac{11}{7}$ , $ \scriptstyle- \frac{4}{5}$ , $ \scriptstyle- \frac{13}{40}$ , $ \scriptstyle8$ , $ \scriptstyle\frac{3}{4}$ , $ \scriptstyle\frac{1}{25}$ , $ \scriptstyle-5$ , $ \scriptstyle-1$ , $ \scriptstyle\frac{5}{2}$ , $ \scriptstyle\frac{3}{8}$ , $ \scriptstyle\frac{17}{3}$ , $ \scriptstyle-7$ , $ \scriptstyle- \frac{21}{65}$ , $ \scriptstyle\frac{4}{3}$ , $ \scriptstyle-1$ , $ \scriptstyle- \frac{7}{10}$ , $ \scriptstyle- \frac{2}{9}$ , $ \scriptstyle- \frac{3}{2}$ , $ \scriptstyle\frac{1}{3}$ , $ \scriptstyle\frac{6}{5}$ , $ \scriptstyle\frac{2}{3}$ , $ \scriptstyle0$ , $ \scriptstyle- \frac{7}{6}$ , $ \scriptstyle\frac{3}{2}$ , $ \scriptstyle\frac{21}{10}$ , $ \scriptstyle\frac{12}{37}$ , $ \scriptstyle-5$ , $ \scriptstyle- \frac{7}{20}$ , $ \scriptstyle- \frac{4}{9}$ , $ \scriptstyle- \frac{3}{2}$ , $ \scriptstyle\frac{41}{15}$ , $ \scriptstyle- \frac{79}{21}$ , $ \scriptstyle- \frac{3}{5}$ , $ \scriptstyle\frac{3}{7}$ , $ \scriptstyle- \frac{2}{5}$ , \\[0.2em] 
$ \scriptstyle- \frac{2}{5}$ , $ \scriptstyle\frac{2}{3}$ , $ \scriptstyle-5$ , $ \scriptstyle- \frac{3}{5}$ , $ \scriptstyle\frac{3}{4}$ , $ \scriptstyle- \frac{7}{10}$ , $ \scriptstyle- \frac{4}{9}$ , $ \scriptstyle- \frac{4}{5}$ , $ \scriptstyle8$ , $ \scriptstyle\frac{3}{5}$ , $ \scriptstyle- \frac{5}{6}$ , $ \scriptstyle- \frac{7}{6}$ , $ \scriptstyle\frac{5}{2}$ , $ \scriptstyle- \frac{2}{9}$ , $ \scriptstyle\frac{2}{3}$ , $ \scriptstyle-5$ , $ \scriptstyle- \frac{3}{7}$ , $ \scriptstyle- \frac{3}{5}$ , $ \scriptstyle\frac{7}{5}$ , $ \scriptstyle\frac{3}{7}$ , $ \scriptstyle\frac{3}{8}$ , $ \scriptstyle\frac{4}{3}$ , $ \scriptstyle7$ , $ \scriptstyle- \frac{3}{2}$ , $ \scriptstyle- \frac{3}{2}$ , $ \scriptstyle3$ , $ \scriptstyle- \frac{1}{2}$ , $ \scriptstyle-1$ , $ \scriptstyle\frac{2}{5}$ , $ \scriptstyle\frac{6}{5}$ , $ \scriptstyle- \frac{2}{5}$ , $ \scriptstyle- \frac{2}{3}$ , $ \scriptstyle0$ , $ \scriptstyle-1$ , $ \scriptstyle\frac{17}{3}$ , $ \scriptstyle\frac{1}{3}$ , $ \scriptstyle- \frac{13}{40}$ , $ \scriptstyle- \frac{16}{3}$ , $ \scriptstyle\frac{41}{15}$ , $ \scriptstyle- \frac{1}{5}$ , $ \scriptstyle-7$ , $ \scriptstyle\frac{11}{7}$ , $ \scriptstyle\frac{3}{2}$ , $ \scriptstyle\frac{7}{2}$ , $ \scriptstyle- \frac{75}{8}$ , $ \scriptstyle\frac{21}{10}$ , $ \scriptstyle\frac{1}{25}$ , $ \scriptstyle- \frac{21}{65}$ , $ \scriptstyle- \frac{7}{20}$ , $ \scriptstyle- \frac{79}{21}$ , $ \scriptstyle\frac{12}{37}$ , $ \scriptstyle- \frac{7}{5}$ , $ \scriptstyle\frac{7}{3}$ , $ \scriptstyle- \frac{19}{3}$ , \end{flushleft} 
    \end{document}