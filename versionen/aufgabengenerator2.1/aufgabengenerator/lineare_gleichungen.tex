\documentclass[12pt]{article}
\usepackage[utf8]{inputenc}
\usepackage{paralist} 
\usepackage{amssymb}  
\usepackage{amsthm}   
\usepackage{multicol} 
\usepackage[left=1.5cm,right=1.5cm,top=0.8cm,bottom=1cm]{geometry} 
\usepackage{amsmath}  
\usepackage{pgf,tikz} 
\usetikzlibrary{arrows} 
\pagestyle{empty} 
\newtheoremstyle{note} 
{16pt}  % Platz zwischen Kopf und voherigem Text 
{16pt}  % und nachfolgendem Text 
{}     % Schriftart des Koerpers 
{}     % mit \parindent Einzug 
{\bf}  % Schriftart des Kopfes 
{:}     % Nach Bedarf z.B. Doppelpunkt nach dem Kopf 
{0.5em} % Platz zwischen Kopf und Koerper 
{}     % Kopfname 

\theoremstyle{note} 
\newtheorem{aufgabe}{Aufgabe} 



\begin{document} 
    \begin{flushleft}
\begin{aufgabe} ~ \\ 
\begin{multicols}{3} 
\begin{enumerate}[a)] 
\item 
$3 x = -7$
\item 
$- 3 x = 2$
\item 
$- 5 x = -4$
\item 
$- 2 x = 1$
\item 
$- 6 x = 0$
\item 
$6 x = 9$
\end{enumerate} 
\end{multicols} 
\end{aufgabe} 
\begin{aufgabe} ~ \\ 
\begin{multicols}{3} 
\begin{enumerate}[a)] 
\item 
$- x = -10$
\item 
$- 3 x = 3$
\item 
$8 x = -5$
\item 
$10 x = -3$
\item 
$- 9 x = 0$
\item 
$- 10 x = -10$
\end{enumerate} 
\end{multicols} 
\end{aufgabe} 
\end{flushleft} 
    \end{document}