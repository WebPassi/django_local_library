\documentclass[12pt,fleqn]{article}
\usepackage[utf8]{inputenc}
\usepackage{paralist} 
\usepackage{amssymb}  
\usepackage{amsthm}   
\usepackage{eurosym} 
\usepackage{multicol} 
\usepackage{framed}
\usepackage[left=1.5cm,right=1.5cm,top=1.5cm,bottom=0.5cm]{geometry} 
\usepackage{fancyheadings} 
\pagestyle{fancy} 
\headheight1.6cm 
\lhead{Name:} 
\chead{\hspace{5cm} Klasse: \qquad\qquad Datum: \qquad\qquad} 
\rhead{\includegraphics[scale=0.3]{/home/pfranz/hbo/logo.png}} 
\lfoot{} 
\cfoot{} 
\rfoot{} 
\usepackage{amsmath}  
\usepackage{cancel} 
\usepackage{pgf,tikz} 
\usetikzlibrary{arrows} 
\newtheoremstyle{aufg} 
{16pt}  % Platz zwischen Kopf und voherigem Text 
{16pt}  % und nachfolgendem Text 
{}     % Schriftart des Koerpers 
{}     % mit \parindent Einzug 
{\bf}  % Schriftart des Kopfes 
{:}     % Nach Bedarf z.B. Doppelpunkt nach dem Kopf 
{0.5em} % Platz zwischen Kopf und Koerper 
{}     % Kopfname 

\theoremstyle{aufg} 
\newtheorem{aufgabe}{Aufgabe} 



\newtheoremstyle{bsp} 
{16pt}  % Platz zwischen Kopf und voherigem Text 
{16pt}  % und nachfolgendem Text 
{}     % Schriftart des Koerpers 
{}     % mit \parindent Einzug 
{\em}  % Schriftart des Kopfes 
{:}     % Nach Bedarf z.B. Doppelpunkt nach dem Kopf 
{0.5em} % Platz zwischen Kopf und Koerper 
{}     % Kopfname 

\theoremstyle{bsp} 
\newtheorem{beispiel}{Beispiel} 


\begin{document} 
    \begin{flushleft}
\begin{center}AB 4 - Zehnerpotenzen und Ma\ss{}einheiten\end{center}\begin{aufgabe} ~ \\ 
Schreibe ohne Zehnerpotenz. \\ 
\begin{multicols}{3} 
\begin{enumerate}[a)] 
\item 
$10^{0}=1$
\item 
$10^{4}=10000$
\item 
$10^{5}=100000$
\item 
$10^{-4}=0.0001$
\item 
$10^{-3}=0.001$
\item 
$10^{-2}=0.01$
\end{enumerate} 
\end{multicols} 
\end{aufgabe} 
\begin{aufgabe} ~ \\ 
Berechne ohne Taschenrechner. \\ 
\begin{multicols}{3} 
\begin{enumerate}[a)] 
\item 
$6\cdot10^{5}=600000$
\item 
$22\cdot10^{0}=22$
\item 
$11\cdot10^{0}=11$
\item 
$0,415\cdot10^{1}=4,15$
\item 
$0,575\cdot10^{3}=575,0$
\item 
$0,851\cdot10^{2}=85,1$
\end{enumerate} 
\end{multicols} 
\end{aufgabe} 
\begin{aufgabe} ~ \\ 
Bringe die Zahlen in die Form $0,... \cdot 10^{?}$. (z.B. $23,46 = 0,2346 \cdot 10^2$) \\ 
\begin{multicols}{3} 
\begin{enumerate}[a)] 
\item 
$0,000322=0,322\cdot10^{-3}$
\item 
$0,00743=0,743\cdot10^{-2}$
\item 
$6970,0=0,697\cdot10^{4}$
\item 
$6130000,0=0,613\cdot10^{7}$
\item 
$0,0325=0,325\cdot10^{-1}$
\item 
$0,0054=0,54\cdot10^{-2}$
\end{enumerate} 
\end{multicols} 
\end{aufgabe} 
\begin{aufgabe} ~ \\ 
Addiere ohne Taschenrechner. \\ 
\begin{multicols}{3} 
\begin{enumerate}[a)] 
\item 
$10^{5}+10^{-3}=100000.001$
\item 
$10^{9}+10^{8}=1100000000$
\item 
$10^{-4}+10^{0}=1.0001$
\end{enumerate} 
\end{multicols} 
\end{aufgabe} 
\begin{aufgabe} ~ \\ 
Multipliziere ohne Taschenrechner. \\ 
\begin{multicols}{2} 
\begin{enumerate}[a)] 
\item 
$1\cdot10^{3}\cdot2\cdot10^{-2}=$$2\cdot 10^{1}=$$20.0$
\item 
$-5\cdot10^{-1}\cdot0\cdot10^{3}=$$0\cdot 10^{2}=$$-0.0$
\item 
$-3\cdot10^{2}\cdot3\cdot10^{5}=$$-9\cdot 10^{7}=$$-90000000$
\item 
$-6\cdot10^{1}\cdot10\cdot10^{1}=$$-60\cdot 10^{2}=$$-6000$
\end{enumerate} 
\end{multicols} 
\end{aufgabe} 
\begin{aufgabe} ~ \\ 
Bringe in die Grundeinheit. \\ 
\begin{multicols}{3} 
\begin{enumerate}[a)] 
\item 
$74\,\mathrm{dl}=74\cdot 10^{-1} \,\mathrm{l^{}}$
\item 
$70\,\mathrm{cl}=70\cdot 10^{-2} \,\mathrm{l^{}}$
\item 
$2\,\mathrm{dm}=2\cdot 10^{-1} \,\mathrm{m^{}}$
\item 
$59\,\mathrm{cm}=59\cdot 10^{-2} \,\mathrm{m^{}}$
\item 
$81\,\mathrm{cm}=81\cdot 10^{-2} \,\mathrm{m^{}}$
\item 
$15\,\mathrm{mg}=15\cdot 10^{-3} \,\mathrm{g^{}}$
\end{enumerate} 
\end{multicols} 
\end{aufgabe} 
\begin{aufgabe} ~ \\ 
Bringe in die Grundeinheit. \\ 
\begin{multicols}{3} 
\begin{enumerate}[a)] 
\item 
$51\,\mathrm{dm^2}=51\cdot 10^{-2} \,\mathrm{m^{2}}$
\item 
$65\,\mathrm{dm^3}=65\cdot 10^{-3} \,\mathrm{m^{3}}$
\item 
$65\,\mathrm{km^3}=65\cdot 10^{9} \,\mathrm{m^{3}}$
\item 
$12\,\mathrm{a}=12\cdot 10^{2} \,\mathrm{m^2} $
\item 
$9\,\mathrm{cm^3}=9\cdot 10^{-6} \,\mathrm{m^{3}}$
\item 
$97\,\mathrm{cm^2}=97\cdot 10^{-4} \,\mathrm{m^{2}}$
\end{enumerate} 
\end{multicols} 
\end{aufgabe} 
\end{flushleft} 
\end{document}