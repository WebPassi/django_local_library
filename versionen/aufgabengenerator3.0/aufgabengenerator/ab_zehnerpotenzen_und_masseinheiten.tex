\documentclass[12pt,fleqn]{article}
\usepackage[utf8]{inputenc}
\usepackage{paralist} 
\usepackage{amssymb}  
\usepackage{amsthm}   
\usepackage{eurosym} 
\usepackage{multicol} 
\usepackage{framed}
\usepackage[left=1.5cm,right=1.5cm,top=1.5cm,bottom=0.5cm]{geometry} 
\usepackage{fancyheadings} 
\pagestyle{fancy} 
\headheight1.6cm 
\lhead{Name:} 
\chead{\hspace{5cm} Klasse: \qquad\qquad Datum: \qquad\qquad} 
\rhead{\includegraphics[scale=0.3]{/home/pfranz/hbo/logo.png}} 
\lfoot{} 
\cfoot{} 
\rfoot{} 
\usepackage{amsmath}  
\usepackage{cancel} 
\usepackage{pgf,tikz} 
\usetikzlibrary{arrows} 
\newtheoremstyle{aufg} 
{16pt}  % Platz zwischen Kopf und voherigem Text 
{16pt}  % und nachfolgendem Text 
{}     % Schriftart des Koerpers 
{}     % mit \parindent Einzug 
{\bf}  % Schriftart des Kopfes 
{:}     % Nach Bedarf z.B. Doppelpunkt nach dem Kopf 
{0.5em} % Platz zwischen Kopf und Koerper 
{}     % Kopfname 

\theoremstyle{aufg} 
\newtheorem{aufgabe}{Aufgabe} 



\newtheoremstyle{bsp} 
{16pt}  % Platz zwischen Kopf und voherigem Text 
{16pt}  % und nachfolgendem Text 
{}     % Schriftart des Koerpers 
{}     % mit \parindent Einzug 
{\em}  % Schriftart des Kopfes 
{:}     % Nach Bedarf z.B. Doppelpunkt nach dem Kopf 
{0.5em} % Platz zwischen Kopf und Koerper 
{}     % Kopfname 

\theoremstyle{bsp} 
\newtheorem{beispiel}{Beispiel} 


\begin{document} 
    \begin{flushleft}
\begin{center}AB 4 - Zehnerpotenzen und Ma\ss{}einheiten\end{center}\begin{aufgabe} ~ \\ 
Schreibe ohne Zehnerpotenz. \\ 
\begin{multicols}{3} 
\begin{enumerate}[a)] 
\item 
$10^{0}=$
\item 
$10^{4}=$
\item 
$10^{5}=$
\item 
$10^{-4}=$
\item 
$10^{-3}=$
\item 
$10^{-2}=$
\end{enumerate} 
\end{multicols} 
\end{aufgabe} 
\begin{aufgabe} ~ \\ 
Berechne ohne Taschenrechner. \\ 
\begin{multicols}{3} 
\begin{enumerate}[a)] 
\item 
$6\cdot10^{5}=$
\item 
$22\cdot10^{0}=$
\item 
$11\cdot10^{0}=$
\item 
$0,415\cdot10^{1}=$
\item 
$0,575\cdot10^{3}=$
\item 
$0,851\cdot10^{2}=$
\end{enumerate} 
\end{multicols} 
\end{aufgabe} 
\begin{aufgabe} ~ \\ 
Bringe die Zahlen in die Form $0,... \cdot 10^{?}$. (z.B. $23,46 = 0,2346 \cdot 10^2$) \\ 
\begin{multicols}{3} 
\begin{enumerate}[a)] 
\item 
$0,000322=$
\item 
$0,00743=$
\item 
$6970,0=$
\item 
$6130000,0=$
\item 
$0,0325=$
\item 
$0,0054=$
\end{enumerate} 
\end{multicols} 
\end{aufgabe} 
\begin{aufgabe} ~ \\ 
Addiere ohne Taschenrechner. \\ 
\begin{multicols}{3} 
\begin{enumerate}[a)] 
\item 
$10^{5}+10^{-3}=$
\item 
$10^{9}+10^{8}=$
\item 
$10^{-4}+10^{0}=$
\end{enumerate} 
\end{multicols} 
\end{aufgabe} 
\begin{aufgabe} ~ \\ 
Multipliziere ohne Taschenrechner. \\ 
\begin{multicols}{2} 
\begin{enumerate}[a)] 
\item 
$1\cdot10^{3}\cdot2\cdot10^{-2}=$
\item 
$-5\cdot10^{-1}\cdot0\cdot10^{3}=$
\item 
$-3\cdot10^{2}\cdot3\cdot10^{5}=$
\item 
$-6\cdot10^{1}\cdot10\cdot10^{1}=$
\end{enumerate} 
\end{multicols} 
\end{aufgabe} 
\begin{aufgabe} ~ \\ 
Bringe in die Grundeinheit. \\ 
\begin{multicols}{3} 
\begin{enumerate}[a)] 
\item 
$74\,\mathrm{dl}=$
\item 
$70\,\mathrm{cl}=$
\item 
$2\,\mathrm{dm}=$
\item 
$59\,\mathrm{cm}=$
\item 
$81\,\mathrm{cm}=$
\item 
$15\,\mathrm{mg}=$
\end{enumerate} 
\end{multicols} 
\end{aufgabe} 
\begin{aufgabe} ~ \\ 
Bringe in die Grundeinheit. \\ 
\begin{multicols}{3} 
\begin{enumerate}[a)] 
\item 
$51\,\mathrm{dm^2}=$
\item 
$65\,\mathrm{dm^3}=$
\item 
$65\,\mathrm{km^3}=$
\item 
$12\,\mathrm{a}=$
\item 
$9\,\mathrm{cm^3}=$
\item 
$97\,\mathrm{cm^2}=$
\end{enumerate} 
\end{multicols} 
\end{aufgabe} 
\end{flushleft} 
\end{document}