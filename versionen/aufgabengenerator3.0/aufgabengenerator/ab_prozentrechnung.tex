\documentclass[12pt,fleqn]{article}
\usepackage[utf8]{inputenc}
\usepackage{paralist} 
\usepackage{amssymb}  
\usepackage{amsthm}   
\usepackage{eurosym} 
\usepackage{multicol} 
\usepackage[left=1.5cm,right=1.5cm,top=1.5cm,bottom=0.5cm]{geometry} 
\usepackage{fancyheadings} 
\pagestyle{fancy} 
\headheight1.6cm 
\lhead{} 
\chead{} 
\rhead{\includegraphics[scale=0.5]{/home/pfranz/hbo/logo.png}} 
\lfoot{} 
\cfoot{} 
\rfoot{} 
\usepackage{amsmath}  
\usepackage{cancel} 
\usepackage{pgf,tikz} 
\usetikzlibrary{arrows} 
\newtheoremstyle{aufg} 
{16pt}  % Platz zwischen Kopf und voherigem Text 
{16pt}  % und nachfolgendem Text 
{}     % Schriftart des Koerpers 
{}     % mit \parindent Einzug 
{\bf}  % Schriftart des Kopfes 
{:}     % Nach Bedarf z.B. Doppelpunkt nach dem Kopf 
{0.5em} % Platz zwischen Kopf und Koerper 
{}     % Kopfname 

\theoremstyle{aufg} 
\newtheorem{aufgabe}{Aufgabe} 



\newtheoremstyle{bsp} 
{16pt}  % Platz zwischen Kopf und voherigem Text 
{16pt}  % und nachfolgendem Text 
{}     % Schriftart des Koerpers 
{}     % mit \parindent Einzug 
{\em}  % Schriftart des Kopfes 
{:}     % Nach Bedarf z.B. Doppelpunkt nach dem Kopf 
{0.5em} % Platz zwischen Kopf und Koerper 
{}     % Kopfname 

\theoremstyle{bsp} 
\newtheorem{beispiel}{Beispiel} 



\begin{document} 
    \begin{flushleft}
Name: \hspace{12cm} Datum:\begin{beispiel} ~ \\ 
Wieviel sind $6\%$ von $5500$ \euro?\qquad {\em L\"osung:} $\frac{6}{100}\cdot 5500$ \euro $ = 330,\hspace{-1pt}00$ \euro\end{beispiel} 
\begin{aufgabe} ~ \\ 
Bestimme den Prozentwert.\begin{multicols}{3} 
\begin{enumerate}[a)] 
\item 
$88\%$ von $2127$ g
\item 
$32\%$ von $4031$ kg
\item 
$91\%$ von $3151$ m
\item 
$72\%$ von $4509$ km
\item 
$66\%$ von $3323$ ml
\item 
$73\%$ von $4636$ t
\item 
$90\%$ von $2170$ \euro~
\item 
$6\%$ von $87$ cm$^2$
\item 
$34\%$ von $1678$ m$^3$
\end{enumerate} 
\end{multicols} 
\end{aufgabe} 
\begin{beispiel} ~ \\ 
Wieviel Prozent sind $2100$ \euro~von $7500$ \euro?\qquad {\em L\"osung:} $\frac{2100}{7500}\cdot 100\%  = 28,\hspace{-1pt}000\%$ \end{beispiel} 
\begin{aufgabe} ~ \\ 
Bestimme den Prozentsatz.\begin{multicols}{3} 
\begin{enumerate}[a)] 
\item 
$2650$ g von $3952$ g
\item 
$2723$ kg von $3014$ kg
\item 
$1619$ m von $4361$ m
\item 
$1307$ km von $4301$ km
\item 
$2850$ ml von $3372$ ml
\item 
$2844$ t von $4567$ t
\item 
$2439$ \euro~ von $3802$ \euro~
\item 
$2232$ cm$^2$ von $4105$ cm$^2$
\item 
$675$ m$^3$ von $3308$ m$^3$
\end{enumerate} 
\end{multicols} 
\end{aufgabe} 
\begin{beispiel} ~ \\ 
$2100$ sind $45\%$. Wieviel sind 100\%?\qquad {\em L\"osung:} $2100$ \euro$\cdot \frac{100}{45}= 4666,\hspace{-1pt}7$ \euro\end{beispiel} 
\begin{aufgabe} ~ \\ 
Bestimme den Grundwert.\begin{multicols}{3} 
\begin{enumerate}[a)] 
\item 
$3019$ g sind $85\%$.
\item 
$1035$ kg sind $59\%$.
\item 
$3374$ m sind $22\%$.
\item 
$3771$ km sind $98\%$.
\item 
$581$ ml sind $82\%$.
\item 
$2948$ t sind $17\%$.
\item 
$4663$ \euro~ sind $32\%$.
\item 
$3997$ cm$^2$ sind $24\%$.
\item 
$4092$ m$^3$ sind $58\%$.
\end{enumerate} 
\end{multicols} 
\end{aufgabe} 
$ \scriptstyle15336,\hspace{-1pt}4$ | $ \scriptstyle3847,\hspace{-1pt}96$ | $ \scriptstyle5,\hspace{-1pt}22000$ | $ \scriptstyle2867,\hspace{-1pt}41$ | $ \scriptstyle708,\hspace{-1pt}537$ | $ \scriptstyle30,\hspace{-1pt}388$ | $ \scriptstyle7055,\hspace{-1pt}17$ | $ \scriptstyle62,\hspace{-1pt}2728$ | $ \scriptstyle1871,\hspace{-1pt}76$ | $ \scriptstyle84,\hspace{-1pt}5196$ | $ \scriptstyle14571,\hspace{-1pt}9$ | $ \scriptstyle17341,\hspace{-1pt}2$ | $ \scriptstyle3246,\hspace{-1pt}48$ | $ \scriptstyle54,\hspace{-1pt}3727$ | $ \scriptstyle37,\hspace{-1pt}1245$ | $ \scriptstyle1289,\hspace{-1pt}92$ | $ \scriptstyle67,\hspace{-1pt}0547$ | $ \scriptstyle570,\hspace{-1pt}520$ | $ \scriptstyle3551,\hspace{-1pt}76$ | $ \scriptstyle16654,\hspace{-1pt}2$ | $ \scriptstyle64,\hspace{-1pt}1505$ | $ \scriptstyle1754,\hspace{-1pt}24$ | $ \scriptstyle90,\hspace{-1pt}345$ | $ \scriptstyle2193,\hspace{-1pt}18$ | $ \scriptstyle1953,\hspace{-1pt}00$ | $ \scriptstyle20,\hspace{-1pt}405$ | $ \scriptstyle3384,\hspace{-1pt}28$ | \end{flushleft} 
    \end{document}