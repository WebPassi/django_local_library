\documentclass[12pt,fleqn]{article}
\usepackage[utf8]{inputenc}
\usepackage{paralist} 
\usepackage{amssymb}  
\usepackage{amsthm}   
\usepackage{eurosym} 
\usepackage{multicol} 
\usepackage[left=1.5cm,right=1.5cm,top=1.5cm,bottom=0.5cm]{geometry} 
\usepackage{fancyheadings} 
\pagestyle{fancy} 
\headheight1.6cm 
\lhead{} 
\chead{} 
\rhead{\includegraphics[scale=0.5]{/home/pfranz/hbo/logo.png}} 
\lfoot{} 
\cfoot{} 
\rfoot{} 
\usepackage{amsmath}  
\usepackage{cancel} 
\usepackage{pgf,tikz} 
\usetikzlibrary{arrows} 
\newtheoremstyle{aufg} 
{16pt}  % Platz zwischen Kopf und voherigem Text 
{16pt}  % und nachfolgendem Text 
{}     % Schriftart des Koerpers 
{}     % mit \parindent Einzug 
{\bf}  % Schriftart des Kopfes 
{:}     % Nach Bedarf z.B. Doppelpunkt nach dem Kopf 
{0.5em} % Platz zwischen Kopf und Koerper 
{}     % Kopfname 

\theoremstyle{aufg} 
\newtheorem{aufgabe}{Aufgabe} 



\newtheoremstyle{bsp} 
{16pt}  % Platz zwischen Kopf und voherigem Text 
{16pt}  % und nachfolgendem Text 
{}     % Schriftart des Koerpers 
{}     % mit \parindent Einzug 
{\em}  % Schriftart des Kopfes 
{:}     % Nach Bedarf z.B. Doppelpunkt nach dem Kopf 
{0.5em} % Platz zwischen Kopf und Koerper 
{}     % Kopfname 

\theoremstyle{bsp} 
\newtheorem{beispiel}{Beispiel} 



\begin{document} 
\begin{flushleft}
{Schriftliche Arbeit zum Erwerb eines Zertifikats im Fach Mathematik} \\ 
{\Large Parabeln und Potenzen}
\hfill Form A \\[2em]
\begin{minipage}[h]{11.5cm}
Name: .............................................................          {\Large Ma 9/10}
\end{minipage}
\begin{minipage}[h]{6cm}
Datum: .................................
\end{minipage} \\[1em]
\renewcommand{\arraystretch}{2.15}
\begin{tabular}{|p{10cm}|p{2cm}|p{2cm}|p{2cm}|}
\hline
\hspace{2cm} Punkte von \qquad 38 \qquad Punkten erreicht & \hspace{1.2cm} NP & G & E \\
\hline
\end{tabular} \\[1em]
{\bf Achtung:} Vergiss nicht, deinen Rechenweg aufzuschreiben und bei den Textaufgaben einen Antwortsatz zu formulieren. \\
\hrulefill \\
{\large \bf Fundamentum} (Dieser Teil muss von {\bf allen} Sch\"ulern bearbeitet werden.) \\

