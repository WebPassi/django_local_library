\documentclass[12pt]{article}
\usepackage[utf8]{inputenc}
\usepackage{paralist} 
\usepackage{amssymb}  
\usepackage{amsthm}   
\usepackage{eurosym} 
\usepackage{multicol} 
\usepackage[left=1.5cm,right=1.5cm,top=1.5cm,bottom=0.5cm]{geometry} 
\usepackage{fancyheadings} 
\pagestyle{fancy} 
\headheight1.6cm 
\lhead{} 
\chead{Prozentrechnung} 
\rhead{\includegraphics[scale=0.5]{/home/pfranz/hbo/logo.png}} 
\lfoot{} 
\cfoot{} 
\rfoot{} 
\usepackage{amsmath}  
\usepackage{cancel} 
\usepackage{pgf,tikz} 
\usetikzlibrary{arrows} 
\newtheoremstyle{aufg} 
{16pt}  % Platz zwischen Kopf und voherigem Text 
{16pt}  % und nachfolgendem Text 
{}     % Schriftart des Koerpers 
{}     % mit \parindent Einzug 
{\bf}  % Schriftart des Kopfes 
{:}     % Nach Bedarf z.B. Doppelpunkt nach dem Kopf 
{0.5em} % Platz zwischen Kopf und Koerper 
{}     % Kopfname 

\theoremstyle{aufg} 
\newtheorem{aufgabe}{Aufgabe} 



\newtheoremstyle{bsp} 
{16pt}  % Platz zwischen Kopf und voherigem Text 
{16pt}  % und nachfolgendem Text 
{}     % Schriftart des Koerpers 
{}     % mit \parindent Einzug 
{\em}  % Schriftart des Kopfes 
{:}     % Nach Bedarf z.B. Doppelpunkt nach dem Kopf 
{0.5em} % Platz zwischen Kopf und Koerper 
{}     % Kopfname 

\theoremstyle{bsp} 
\newtheorem{beispiel}{Beispiel} 



\begin{document} 
    \begin{flushleft}
Name: \hspace{12cm} Datum:\begin{aufgabe} ~ \\ 
\begin{multicols}{3} 
\begin{enumerate}[a)] 
\item 
- 2 x = 2
\item 
5 x = 3
\item 
2 x = 10
\item 
7 x = 3
\item 
2 x = -2
\item 
x = -7
\item 
- 9 x = 4
\item 
- 4 x = 1
\item 
3 x = 2
\end{enumerate} 
\end{multicols} 
\end{aufgabe} 
\begin{aufgabe} ~ \\ 
\begin{multicols}{3} 
\begin{enumerate}[a)] 
\item 
6 x - 10 = 0
\item 
- 2 x - 5 = 0
\item 
5 x - 9 = 0
\item 
- 8 x + 9 = 0
\item 
- 3 x + 9 = 0
\item 
3 x - 8 = 0
\item 
10 x - 3 = 0
\item 
- 8 x + 9 = 0
\item 
- 10 x - 9 = 0
\end{enumerate} 
\end{multicols} 
\end{aufgabe} 
\begin{aufgabe} ~ \\ 
\begin{multicols}{3} 
\begin{enumerate}[a)] 
\item 
4 x + 2 = -6
\item 
8 x + 1 = -6
\item 
x - 7 = 2
\item 
7 x - 4 = 8
\item 
6 x + 6 = 9
\item 
- 6 x - 10 = -3
\item 
- 10 x - 1 = 8
\item 
- 5 x + 10 = 6
\item 
2 x = 3
\end{enumerate} 
\end{multicols} 
\end{aufgabe} 
\begin{aufgabe} ~ \\ 
\begin{multicols}{3} 
\begin{enumerate}[a)] 
\item 
- 9 x - 6 = - 6 x + 3
\item 
- 4 x + 3 = 10 x + 8
\item 
- 10 x - 6 = x + 5
\item 
- 7 x + 9 = 5 x + 10
\item 
- 4 x - 1 = x - 4
\item 
9 x - 9 = 4 x - 2
\item 
7 x + 4 = x - 9
\item 
8 x + 8 = 8 x + 6
\item 
- 2 x + 10 = - 9 x - 2
\end{enumerate} 
\end{multicols} 
\end{aufgabe} 
\begin{aufgabe} ~ \\ 
\begin{multicols}{3} 
\begin{enumerate}[a)] 
\item 
-1(9 x - 5) = -6
\item 
-4(- 4 x - 4) = -10
\item 
6(3 x + 6) = -3
\item 
6(10 x - 4) = 0
\item 
4(x - 4) = -6
\item 
4(4 x + 5) = 3
\item 
5(- 3 x - 3) = -2
\item 
3(- 5 x + 4) = 1
\item 
- x = 1
\end{enumerate} 
\end{multicols} 
\end{aufgabe} 
\begin{aufgabe} ~ \\ 
\begin{multicols}{2} 
\begin{enumerate}[a)] 
\item 
-2(4 x) = -4(10 x - 2)
\item 
5(2 x - 2) = -10(6 x + 6)
\item 
2(x + 9) = 10(9 x - 2)
\item 
x + 2 = 8(4 x - 6)
\item 
- 9 x + 2 = -7(- 7 x - 3)
\item 
-3(- 6 x + 9) = -10(- 9 x - 8)
\item 
9(5 x + 8) = 8(2 x + 4)
\item 
4(- 4 x - 1) = 3(- 5 x + 7)
\item 
7(- 9 x + 2) = 10(8 x - 3)
\end{enumerate} 
\end{multicols} 
\end{aufgabe} 
$ \scriptstyle- \frac{1}{12}$ , $ \scriptstyle- \frac{4}{9}$ , $ \scriptstyle\frac{19}{44}$ , $ \scriptstyle\frac{3}{2}$ , $ \scriptstyle- \frac{13}{6}$ , $ \scriptstyle\frac{2}{5}$ , $ \scriptstyle- \frac{5}{7}$ , $ \scriptstyle- \frac{5}{14}$ , $ \scriptstyle\frac{3}{5}$ , $ \scriptstyle- \frac{7}{6}$ , $ \scriptstyle\frac{11}{15}$ , $ \scriptstyle\frac{11}{9}$ , $ \scriptstyle- \frac{19}{58}$ , $ \scriptstyle\frac{3}{7}$ , $ \scriptstyle-1$ , $ \scriptstyle-1$ , $ \scriptstyle- \frac{9}{10}$ , $ \scriptstyle- \frac{13}{6}$ , $ \scriptstyle-7$ , $ \scriptstyle\frac{5}{2}$ , $ \scriptstyle\frac{50}{31}$ , $ \scriptstyle\frac{9}{8}$ , $ \scriptstyle-3$ , $ \scriptstyle3$ , $ \scriptstyle- \frac{13}{15}$ , $ \scriptstyle9$ , $ \scriptstyle- \frac{17}{16}$ , $ \scriptstyle- \frac{7}{8}$ , $ \scriptstyle-1$ , $ \scriptstyle\frac{4}{13}$ , $ \scriptstyle\frac{3}{10}$ , $ \scriptstyle-25$ , $ \scriptstyle\frac{9}{5}$ , $ \scriptstyle-2$ , $ \scriptstyle\frac{9}{8}$ , $ \scriptstyle\frac{12}{7}$ , $ \scriptstyle-1$ , $ \scriptstyle- \frac{13}{8}$ , $ \scriptstyle\frac{5}{3}$ , $ \scriptstyle- \frac{40}{29}$ , $ \scriptstyle- \frac{9}{10}$ , $ \scriptstyle\frac{1}{2}$ , $ \scriptstyle\frac{3}{5}$ , $ \scriptstyle\frac{1}{4}$ , $ \scriptstyle- \frac{5}{2}$ , $ \scriptstyle\frac{7}{5}$ , $ \scriptstyle- \frac{107}{72}$ , $ \scriptstyle\frac{2}{3}$ , $ \scriptstyle- \frac{1}{4}$ , $ \scriptstyle\frac{4}{5}$ , $ \scriptstyle5$ , $ \scriptstyle- \frac{12}{7}$ , $ \scriptstyle\frac{8}{3}$ , $ \scriptstyle\{\}$ , \\[0.2em] 
$ \scriptstyle-1$ , $ \scriptstyle\frac{3}{5}$ , $ \scriptstyle5$ , $ \scriptstyle\frac{3}{7}$ , $ \scriptstyle-1$ , $ \scriptstyle-7$ , $ \scriptstyle- \frac{4}{9}$ , $ \scriptstyle- \frac{1}{4}$ , $ \scriptstyle\frac{2}{3}$ , $ \scriptstyle\frac{5}{3}$ , $ \scriptstyle- \frac{5}{2}$ , $ \scriptstyle\frac{9}{5}$ , $ \scriptstyle\frac{9}{8}$ , $ \scriptstyle3$ , $ \scriptstyle\frac{8}{3}$ , $ \scriptstyle\frac{3}{10}$ , $ \scriptstyle\frac{9}{8}$ , $ \scriptstyle- \frac{9}{10}$ , $ \scriptstyle-2$ , $ \scriptstyle- \frac{7}{8}$ , $ \scriptstyle9$ , $ \scriptstyle\frac{12}{7}$ , $ \scriptstyle\frac{1}{2}$ , $ \scriptstyle- \frac{7}{6}$ , $ \scriptstyle- \frac{9}{10}$ , $ \scriptstyle\frac{4}{5}$ , $ \scriptstyle\frac{3}{2}$ , $ \scriptstyle-3$ , $ \scriptstyle- \frac{5}{14}$ , $ \scriptstyle-1$ , $ \scriptstyle- \frac{1}{12}$ , $ \scriptstyle\frac{3}{5}$ , $ \scriptstyle\frac{7}{5}$ , $ \scriptstyle- \frac{13}{6}$ , $ \scriptstyle\{\}$ , $ \scriptstyle- \frac{12}{7}$ , $ \scriptstyle\frac{11}{9}$ , $ \scriptstyle- \frac{13}{8}$ , $ \scriptstyle- \frac{13}{6}$ , $ \scriptstyle\frac{2}{5}$ , $ \scriptstyle\frac{5}{2}$ , $ \scriptstyle- \frac{17}{16}$ , $ \scriptstyle- \frac{13}{15}$ , $ \scriptstyle\frac{11}{15}$ , $ \scriptstyle-1$ , $ \scriptstyle\frac{1}{4}$ , $ \scriptstyle- \frac{5}{7}$ , $ \scriptstyle\frac{19}{44}$ , $ \scriptstyle\frac{50}{31}$ , $ \scriptstyle- \frac{19}{58}$ , $ \scriptstyle- \frac{107}{72}$ , $ \scriptstyle- \frac{40}{29}$ , $ \scriptstyle-25$ , $ \scriptstyle\frac{4}{13}$ , \end{flushleft} 
    \end{document}