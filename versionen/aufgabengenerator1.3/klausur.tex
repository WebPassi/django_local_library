% basic
\documentclass[fleqn,leqno,11pt]{scrartcl} 
\usepackage{multicol}
\usepackage[ngerman]{babel}
\usepackage[utf8]{inputenc}
\usepackage{pgfplots}
\usepackage{fancyhdr}
\usetikzlibrary{arrows}
\usepackage {graphicx}
\usepackage{paralist} % enumerate
\usepackage{amsthm}   % \newtheorem-Umgebung
\usepackage{amsmath}  % \align-Umgebung
\usepackage{amssymb}  % z.B. fuer reelles Zahlensymbol
\usepackage{framed}
% Dimension
\usepackage[left=1.5cm,right=1.5cm,top=1.5cm,bottom=1cm]{geometry}


% Neue Kommandos
\newcommand{\lack}{\hspace{3cm}}
\newcommand{\Vekz}[2]{\left(\begin{array}{r} #1 \\ #2 \end{array}\right)}
\newcommand{\Vekd}[3]{\left(\begin{array}{r} #1 \\ #2 \\ #3 \end{array}\right)}

% Umgebung outfits
\newtheoremstyle{aufgabenstyle}
{16pt}  % Platz zwischen Kopf und voherigem Text
{16pt}  % und nachfolgendem Text
{}     % Schriftart des Koerpers
{}     % mit \parindent Einzug
{\bf}  % Schriftart des Kopfes
{:}     % Nach Bedarf z.B. Doppelpunkt nach dem Kopf
{0.5em} % Platz zwischen Kopf und Koerper
{}     % Kopfname

% Umgebungsnamen
\theoremstyle{aufgabenstyle}
\newtheorem{aufgabe}{Aufgabe}

% Pagestyle Header and co
\pagestyle{fancy}
\lhead{\includegraphics[scale=0.25]{logo14_15.png}}
\chead{\hspace{0cm} {Klausur ma - 2 (Frz)}}
\rhead{05.03.15}
\lfoot{}
\cfoot{}
\rfoot{}




\begin{document}

\begin{flushleft}
Name: \\[2em]
\begin{center}
{\Large Thema: Integralrechnung}
\end{center}
\vspace{2em}


\begin{aufgabe}[3+4={\bf 7 Punkte}] ~ \\
Berechnen Sie die bestimmten Integrale.
\begin{multicols}{2}
\begin{enumerate}[a)]
\item $\int\limits_{1}^{3} x^{3}\, dx$
\item $\int\limits_{-2}^{2} x^{7}-2x^{3}\, dx$
\end{enumerate}
\end{multicols}
\end{aufgabe}


\begin{aufgabe}[2+4+4+4+1={\bf 15 Punkte}] ~ \\
Gegeben ist die Funktion
\[ f(x)=\mathrm{e}^{\frac{1}{4}x}. \]
\begin{enumerate}[a)]
\item Bestimmen Sie die Funktionswerte $f(-4)$, $f(0)$, $f(4)$ und $f(8)$.
\item Zeichnen Sie den Graphen von $f$ im Intervall $[-4;8]$. 
\item Bestimmen Sie n\"aherungsweise den Fl\"acheninhalt zwischen dem Graphen von $f$ und der $x$-Achse \"uber dem Intervall $[0;8]$. Nutzen Sie dazu die Untersumme $U_2$ und veranschaulichen Sie Ihre Vorgehenseise an Ihrer Zeichnung.
\item Eine Stammfunktion von $f$ lautet:
\[ F(x)=4\mathrm{e}^{\frac{1}{4}x}. \]
Bestimmen Sie damit den exakten Fl\"acheninhalt zwischen Graph und $x$-Achse \"uber dem Intervall $[0;8]$ und geben Sie an, welchen prozentualen Fehler Ihre N\"aherungsl\"osung hat.
\item Beschreiben Sie eine M\"oglichkeit, wie Sie den Fehler verringern k\"onnen.
\end{enumerate}
\end{aufgabe}


\begin{aufgabe}[{\bf 8} oder {\bf 12 Punkte}] ~ \\
Der Graph von $f$ und die $x$-Achse schlie\ss{}en eine Fl\"ache ein. Bestimmen Sie deren Inhalt. Verwenden Sie f\"ur $f$ \\ 
{\bf entweder}
\[ f(x)=x^2+4x-5 \]
{\bf oder}
\[ f(x)=-2x^3+6x^2+20x \].
\end{aufgabe}

\begin{aufgabe}[{\bf 15 Punkte}] ~ \\
\begin{minipage}{0.45\textwidth}
Zwei Seen sind durch einen 8 Meter breiten Kanal verbunden. An der tiefsten Stelle hat der Kanal eine Tiefe von 3 Meter. Der Querschnitt des Kanals ist auf dem Bild rechts (nicht ma\ss{}stabsgetreu) abgebildet. Bestimmen Sie, wieviel Wasser der Kanal aufnehmen kann, wenn er eine L\"ange von 500 Meter besitzt. 
\end{minipage}
\begin{minipage}{0.1\textwidth}
~
\end{minipage}
\begin{minipage}{0.35\textwidth}
\includegraphics[scale=0.65]{kanal.png}
\end{minipage}
\end{aufgabe}
\end{flushleft}
\end{document}


