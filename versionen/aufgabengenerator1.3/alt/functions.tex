\documentclass[12pt]{article}
\usepackage[utf8]{inputenc}
\usepackage{paralist} 
\usepackage{amssymb}  
\usepackage{amsthm}   
\usepackage{multicol} 
\usepackage[left=1.5cm,right=1.5cm,top=0.8cm,bottom=1cm]{geometry} 
\usepackage{amsmath}  
\usepackage{pgf,tikz} 
\usetikzlibrary{arrows} 
\pagestyle{empty} 
\newtheoremstyle{note} 
{16pt}  % Platz zwischen Kopf und voherigem Text 
{16pt}  % und nachfolgendem Text 
{}     % Schriftart des Koerpers 
{}     % mit \parindent Einzug 
{\bf}  % Schriftart des Kopfes 
{:}     % Nach Bedarf z.B. Doppelpunkt nach dem Kopf 
{0.5em} % Platz zwischen Kopf und Koerper 
{}     % Kopfname 

\theoremstyle{note} 
\newtheorem{aufgabe}{Aufgabe} 



\begin{document}
\begin{flushleft}
$f(x)=-2x+-2$ \\ 
$f(x)=\frac{1}{2}x+-1$ \\ 
$f(x)=1x+0$ \\ 
$f(x)=\frac{2}{3}x+1$ \\ 
$f(x)=0x+0$ \\ 
$f(x)=\frac{1}{3}x+3$ \\ 
$f(x)=2x+0$ \\ 
$f(x)=\frac{1}{2}x+-3$ \\ 
$f(x)=1x+2$ \\ 
$f(x)=- \frac{1}{3}x+1$ \\ 
$f(x)=- \frac{2}{3}x+1$ \\ 
$f(x)=- \frac{3}{2}x+0$ \\ 
$f(x)=\frac{1}{3}x+-3$ \\ 
$f(x)=0x+-3$ \\ 
$f(x)=\frac{2}{3}x+-2$ \\ 
\end{flushleft}
\end{document}
