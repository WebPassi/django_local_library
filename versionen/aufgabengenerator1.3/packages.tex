% basic
\documentclass[fleqn,leqno,11pt]{scrartcl} 
\usepackage{multicol}
\usepackage[ngerman]{babel}
\usepackage[utf8]{inputenc}
\usepackage{pgfplots}
\usepackage{fancyhdr}
\usetikzlibrary{arrows}
\usepackage {graphicx}
\usepackage{paralist} % enumerate
\usepackage{amsthm}   % \newtheorem-Umgebung
\usepackage{amsmath}  % \align-Umgebung
\usepackage{amssymb}  % z.B. fuer reelles Zahlensymbol
\usepackage{framed}
% Dimension
\usepackage[left=1.5cm,right=1.5cm,top=1.5cm,bottom=1cm]{geometry}


% Neue Kommandos
\newcommand{\lack}{\hspace{3cm}}
\newcommand{\Vekz}[2]{\left(\begin{array}{r} #1 \\ #2 \end{array}\right)}
\newcommand{\Vekd}[3]{\left(\begin{array}{r} #1 \\ #2 \\ #3 \end{array}\right)}

% Umgebung outfits
\newtheoremstyle{aufgabenstyle}
{16pt}  % Platz zwischen Kopf und voherigem Text
{16pt}  % und nachfolgendem Text
{}     % Schriftart des Koerpers
{}     % mit \parindent Einzug
{\bf}  % Schriftart des Kopfes
{:}     % Nach Bedarf z.B. Doppelpunkt nach dem Kopf
{0.5em} % Platz zwischen Kopf und Koerper
{}     % Kopfname

% Umgebungsnamen
\theoremstyle{aufgabenstyle}
\newtheorem{aufgabe}{Aufgabe}

% Pagestyle Header and co
\pagestyle{fancy}
\lhead{\includegraphics[scale=0.25]{logo14_15.png}}
\chead{\hspace{0cm} {Klausur ma - 2 (Frz)}}
\rhead{05.03.15}
\lfoot{}
\cfoot{}
\rfoot{}




