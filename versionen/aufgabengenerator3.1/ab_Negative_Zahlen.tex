\documentclass[12pt,fleqn]{article}
\usepackage[utf8]{inputenc}
\usepackage{paralist} 
\usepackage{amssymb}  
\usepackage{amsthm}   
\usepackage{eurosym} 
\usepackage{multicol} 
\usepackage{framed}
\usepackage[left=1.5cm,right=1.5cm,top=1.5cm,bottom=0.5cm]{geometry} 
\usepackage{fancyheadings} 
\pagestyle{fancy} 
\headheight1.6cm 
\lhead{Name:} 
\chead{\hspace{5cm} Klasse: \qquad\qquad Datum: \qquad\qquad} 
\rhead{\includegraphics[scale=0.3]{/home/pfranz/hbo/logo.png}} 
\lfoot{} 
\cfoot{} 
\rfoot{} 
\usepackage{amsmath}  
\usepackage{cancel} 
\usepackage{pgf,tikz} 
\usetikzlibrary{arrows} 
\newtheoremstyle{aufg} 
{16pt}  % Platz zwischen Kopf und voherigem Text 
{16pt}  % und nachfolgendem Text 
{}     % Schriftart des Koerpers 
{}     % mit \parindent Einzug 
{\bf}  % Schriftart des Kopfes 
{:}     % Nach Bedarf z.B. Doppelpunkt nach dem Kopf 
{0.5em} % Platz zwischen Kopf und Koerper 
{}     % Kopfname 

\theoremstyle{aufg} 
\newtheorem{aufgabe}{Aufgabe} 



\newtheoremstyle{bsp} 
{16pt}  % Platz zwischen Kopf und voherigem Text 
{16pt}  % und nachfolgendem Text 
{}     % Schriftart des Koerpers 
{}     % mit \parindent Einzug 
{\em}  % Schriftart des Kopfes 
{:}     % Nach Bedarf z.B. Doppelpunkt nach dem Kopf 
{0.5em} % Platz zwischen Kopf und Koerper 
{}     % Kopfname 

\theoremstyle{bsp} 
\newtheorem{beispiel}{Beispiel} 


\begin{document} 
    \begin{flushleft}
Name: \hspace{12cm} Datum:\begin{aufgabe} ~ \\ 
\begin{multicols}{3} 
\begin{enumerate}[a)] 
\item 
$7+(-7)=$
\item 
$9+1=$
\item 
$-6+(-5)=$
\item 
$-8+(-7)=$
\item 
$4+(-3)=$
\item 
$-6+3=$
\item 
$6+(-9)=$
\item 
$-9+(-7)=$
\item 
$-3+6=$
\end{enumerate} 
\end{multicols} 
\end{aufgabe} 
\begin{aufgabe} ~ \\ 
\begin{multicols}{3} 
\begin{enumerate}[a)] 
\item 
$-1-7=$
\item 
$-1-1=$
\item 
$-6-1=$
\item 
$4-1=$
\item 
$8-9=$
\item 
$9-6=$
\item 
$1-3=$
\item 
$-9-(-2)=$
\item 
$-3-(-2)=$
\end{enumerate} 
\end{multicols} 
\end{aufgabe} 
\begin{aufgabe} ~ \\ 
\begin{multicols}{3} 
\begin{enumerate}[a)] 
\item 
$-6+6+1=$
\item 
$2+1+(-1)=$
\item 
$-6+1+(-7)=$
\end{enumerate} 
\end{multicols} 
\end{aufgabe} 
\begin{aufgabe} ~ \\ 
\begin{multicols}{3} 
\begin{enumerate}[a)] 
\item 
$2+(-4)-9=$
\item 
$1+3-(-3)=$
\item 
$3+(-2)-3=$
\item 
$-6+(-3)-9=$
\item 
$-4+7-5=$
\item 
$4+(-7)-(-1)=$
\item 
$1+(-8)-(-3)=$
\item 
$-7+4-1=$
\item 
$3+(-1)-(-6)=$
\end{enumerate} 
\end{multicols} 
\end{aufgabe} 
\begin{aufgabe} ~ \\ 
\begin{multicols}{3} 
\begin{enumerate}[a)] 
\item 
$9-9+4=$
\item 
$-9-1+9=$
\item 
$2-(-1)+2=$
\item 
$6-5+1=$
\item 
$-9-(-7)+5=$
\item 
$-8-1+(-6)=$
\item 
$1-5+1=$
\item 
$9-5+(-4)=$
\item 
$-6-2+(-2)=$
\end{enumerate} 
\end{multicols} 
\end{aufgabe} 
\begin{aufgabe} ~ \\ 
\begin{multicols}{3} 
\begin{enumerate}[a)] 
\item 
$-6-4-3=$
\item 
$-1-(-4)-7=$
\item 
$-4-2-4=$
\item 
$9-(-7)-3=$
\item 
$1-3-4=$
\item 
$5-(-6)-(-3)=$
\item 
$6-(-5)-1=$
\item 
$-6-(-1)-(-7)=$
\item 
$-6-3-6=$
\end{enumerate} 
\end{multicols} 
\end{aufgabe} 
\end{flushleft} 
\end{document}