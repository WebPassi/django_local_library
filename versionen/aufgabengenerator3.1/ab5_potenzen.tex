\documentclass[12pt,fleqn]{article}
\usepackage[utf8]{inputenc}
\usepackage{paralist} 
\usepackage{amssymb}  
\usepackage{amsthm}   
\usepackage{eurosym} 
\usepackage{multicol} 
\usepackage{framed}
\usepackage[left=1.5cm,right=1.5cm,top=1.5cm,bottom=0.5cm]{geometry} 
\usepackage{fancyheadings} 
\pagestyle{fancy} 
\headheight1.6cm 
\lhead{Name:} 
\chead{\hspace{5cm} Klasse: \qquad\qquad Datum: \qquad\qquad} 
\rhead{\includegraphics[scale=0.3]{/home/pfranz/hbo/logo.png}} 
\lfoot{} 
\cfoot{} 
\rfoot{} 
\usepackage{amsmath}  
\usepackage{cancel} 
\usepackage{pgf,tikz} 
\usetikzlibrary{arrows} 
\newtheoremstyle{aufg} 
{16pt}  % Platz zwischen Kopf und voherigem Text 
{16pt}  % und nachfolgendem Text 
{}     % Schriftart des Koerpers 
{}     % mit \parindent Einzug 
{\bf}  % Schriftart des Kopfes 
{:}     % Nach Bedarf z.B. Doppelpunkt nach dem Kopf 
{0.5em} % Platz zwischen Kopf und Koerper 
{}     % Kopfname 

\theoremstyle{aufg} 
\newtheorem{aufgabe}{Aufgabe} 



\newtheoremstyle{bsp} 
{16pt}  % Platz zwischen Kopf und voherigem Text 
{16pt}  % und nachfolgendem Text 
{}     % Schriftart des Koerpers 
{}     % mit \parindent Einzug 
{\em}  % Schriftart des Kopfes 
{:}     % Nach Bedarf z.B. Doppelpunkt nach dem Kopf 
{0.5em} % Platz zwischen Kopf und Koerper 
{}     % Kopfname 

\theoremstyle{bsp} 
\newtheorem{beispiel}{Beispiel} 


\begin{document} 
    \begin{flushleft}
\begin{center}AB 5 - Potenzen\end{center}\begin{aufgabe} ~ \\ 
Schreibe ohne Potenz als Multiplikationsaufgabe. \\ 
\begin{multicols}{3} 
\begin{enumerate}[a)] 
\item 
$5^{7}=$
\item 
$6^{2}=$
\item 
$3^{5}=$
\end{enumerate} 
\end{multicols} 
\end{aufgabe} 
\begin{aufgabe} ~ \\ 
Schreibe als Potenz. \\ 
\begin{multicols}{3} 
\begin{enumerate}[a)] 
\item 
$2\cdot2\cdot2\cdot2=$
\item 
$7\cdot7=$
\item 
$5\cdot5\cdot5\cdot5=$
\end{enumerate} 
\end{multicols} 
\end{aufgabe} 
\begin{aufgabe} ~ \\ 
Schreibe als Bruch. \\ 
\begin{multicols}{3} 
\begin{enumerate}[a)] 
\item 
$6^{-3}=$
\item 
$4^{-8}=$
\item 
$9^{-4}=$
\end{enumerate} 
\end{multicols} 
\end{aufgabe} 
\begin{aufgabe} ~ \\ 
Schreibe als Wurzel. \\ 
\begin{multicols}{3} 
\begin{enumerate}[a)] 
\item 
$5^{\frac{8}{11}}=$
\item 
$4^{\frac{7}{3}}=$
\item 
$3^{\frac{8}{7}}=$
\end{enumerate} 
\end{multicols} 
\end{aufgabe} 
\begin{aufgabe} ~ \\ 
Schreibe als Potenz. \\ 
\begin{multicols}{3} 
\begin{enumerate}[a)] 
\item 
$\sqrt[14]{7}^{\,11}=$
\item 
$\sqrt[2]{2}^{\,7}=$
\item 
$\sqrt[5]{5}^{\,14}=$
\end{enumerate} 
\end{multicols} 
\end{aufgabe} 
\begin{aufgabe} ~ \\ 
Schreibe als eine Potenz. \\ 
\begin{multicols}{3} 
\begin{enumerate}[a)] 
\item 
$3^{-5}\cdot3^{0}=$
\item 
$9^{8}\cdot9^{2}=$
\item 
$6^{-4}\cdot6^{-4}=$
\end{enumerate} 
\end{multicols} 
\end{aufgabe} 
\begin{aufgabe} ~ \\ 
Schreibe als eine Potenz. \\ 
\begin{multicols}{3} 
\begin{enumerate}[a)] 
\item 
$\frac{2^{-10}}{2^{-8}}=$
\item 
$\frac{2^{-5}}{2^{-5}}=$
\item 
$\frac{6^{7}}{6^{4}}=$
\end{enumerate} 
\end{multicols} 
\end{aufgabe} 
\begin{aufgabe} ~ \\ 
Berechne mit dem Taschenrechner. \\ 
\begin{multicols}{2} 
\begin{enumerate}[a)] 
\item 
$1\cdot5^{5}+6\cdot8^{-2}=$
\item 
$-5\cdot3^{-1}+4\cdot4^{2}=$
\item 
$-4\cdot7^{5}+10\cdot6^{-2}=$
\item 
$-6\cdot6^{-2}+1\cdot8^{-1}=$
\end{enumerate} 
\end{multicols} 
\end{aufgabe} 
\end{flushleft} 
\end{document}