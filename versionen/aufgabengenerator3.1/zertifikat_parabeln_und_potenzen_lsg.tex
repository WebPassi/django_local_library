\documentclass[12pt,fleqn]{article}
\usepackage[utf8]{inputenc}
\usepackage{paralist} 
\usepackage{amssymb}  
\usepackage{amsthm}   
\usepackage{eurosym} 
\usepackage{multicol} 
\usepackage{framed}
\usepackage[left=1.5cm,right=1.5cm,top=1.5cm,bottom=0.5cm]{geometry} 
\usepackage{fancyheadings} 
\pagestyle{fancy} 
\headheight1.6cm 
\lhead{Name:} 
\chead{\hspace{5cm} Klasse: \qquad\qquad Datum: \qquad\qquad} 
\rhead{\includegraphics[scale=0.3]{/home/pfranz/hbo/logo.png}} 
\lfoot{} 
\cfoot{} 
\rfoot{} 
\usepackage{amsmath}  
\usepackage{cancel} 
\usepackage{pgf,tikz} 
\usetikzlibrary{arrows} 
\newtheoremstyle{aufg} 
{16pt}  % Platz zwischen Kopf und voherigem Text 
{16pt}  % und nachfolgendem Text 
{}     % Schriftart des Koerpers 
{}     % mit \parindent Einzug 
{\bf}  % Schriftart des Kopfes 
{:}     % Nach Bedarf z.B. Doppelpunkt nach dem Kopf 
{0.5em} % Platz zwischen Kopf und Koerper 
{}     % Kopfname 

\theoremstyle{aufg} 
\newtheorem{aufgabe}{Aufgabe} 



\newtheoremstyle{bsp} 
{16pt}  % Platz zwischen Kopf und voherigem Text 
{16pt}  % und nachfolgendem Text 
{}     % Schriftart des Koerpers 
{}     % mit \parindent Einzug 
{\em}  % Schriftart des Kopfes 
{:}     % Nach Bedarf z.B. Doppelpunkt nach dem Kopf 
{0.5em} % Platz zwischen Kopf und Koerper 
{}     % Kopfname 

\theoremstyle{bsp} 
\newtheorem{beispiel}{Beispiel} 


\begin{document} 
    \begin{flushleft}
\begin{center}{Schriftliche Arbeit zum Erwerb eines Zertifikats im Fach Mathematik}\end{center} 
\begin{center}{\Large Parabeln und Potenzen}\end{center} 
\renewcommand{\arraystretch}{2.15} 
\begin{tabular}{|p{10cm}|p{2cm}|p{2cm}|p{2cm}|} 
\hline 
\hspace{2cm} Punkte von \qquad32\qquad Punkten erreicht & \hspace{1.2cm} NP & G & E \\ 
\hline 
\end{tabular} \\[1em]    
{\bf Hilfsmittel: }nicht programmierf\"ahiger Taschenrechner, Formelsammlung\\ 
{\bf Zeit: }45 Minuten\\ 
\begin{center} {\bf Bearbeite so viele Aufgaben wie m\"oglich.} \end{center}\begin{center} \begin{framed} Grundniveau (9 Punkte) \end{framed} \end{center}\begin{aufgabe}\framebox{\qquad/6} ~ \\ 
Gib f\"ur die folgenden Parabeln den Scheitelpunkt, die Symmetrieachse, sowie die Nullstellen an. 
\begin{minipage}{0.25\textwidth} 
\definecolor{cqcqcq}{rgb}{0.75,0.75,0.75} 
\begin{tikzpicture}[domain=-3:3,scale=0.7][line cap=round,line join=round,>=triangle 45,x=1.0cm,y=1.0cm]\draw[color=cqcqcq,dash pattern=on 2pt off 2pt, xstep=1.0cm,ystep=1.0cm] (-3.0,-3.0) grid (3.0,3.0); 
\draw[->] (-3.0,0) -- (3.0,0) node[below] {\footnotesize $x$}; 
\foreach \x in {-3, -2, -1,  1, 2}
\draw[shift={(\x,0)},color=black] (0pt,2pt) -- (0pt,-2pt) node[below] {\footnotesize $\x$};\draw[->,color=black] (0,-3.0) -- (0,3.0) node[right] {\footnotesize $y$}; 
\foreach \y in {-3, -2, -1,  1, 2}
\draw[shift={(0,\y)},color=black] (2pt,0pt) -- (-2pt,0pt) node[left] {\footnotesize $\y$}; 
\draw[color=black] (0pt,-10pt) node[right] {\footnotesize $0$}; 
\begin{scope} 
\clip (-3,-3) rectangle (3,3); 
\draw[color=black] plot[smooth] function{x**2 - 2}; 
\draw (0.5,-1.8) node[right] {$\scriptscriptstyle f_1$}; 
\draw[color=black] plot[smooth] function{x**2 - 2}; 
\draw (0.5,-1.8) node[right] {$\scriptscriptstyle f_2$}; 
\end{scope} 
\end{tikzpicture}\end{minipage} 
\begin{minipage}{0.1\textwidth} 
 ~ \end{minipage} 
\begin{minipage}{0.55\textwidth} 
\renewcommand{\arraystretch}{1.3} 
\begin{tabular}{c|c|c|c}
 & Scheitelpunkt & Symmetrieachse & Nullstellen\\ \hline 
$f_1$ & $S(0|-2)$ & $x=0$ & $x_1\approx -1,\hspace{-1pt}41\quad x_2\approx1,\hspace{-1pt}41$\\ \hline 
$f_2$ & $S(0|-2)$ & $x=0$ & $x_1\approx -1,\hspace{-1pt}41\quad x_2\approx1,\hspace{-1pt}41$\\ 

\end{tabular} 

\end{minipage} 

\end{aufgabe} 
\begin{aufgabe}\framebox{\qquad/3} ~ \\ 
Berechne ohne Taschenrechner. \\ 
\begin{multicols}{3} 
\begin{enumerate}[a)] 
\item 
$20\cdot10^{2}=2000$
\item 
$10\cdot10^{-4}=0,001$
\item 
$-3\cdot10^{-3}=-0,003$
\end{enumerate} 
\end{multicols} 
\end{aufgabe} 

 \clearpage 
\begin{center} \begin{framed} Standardniveau (19 Punkte) \end{framed} \end{center}\begin{aufgabe}\framebox{\qquad/7} ~ \\ 
Erstelle f\"ur die Funktion $f(x)=x^{2} - x + 3$ eine Wertetabelle und zeichne den dazugeh\"origen Graphen im Bereich von $x=-3$ bis $x=3$ in ein Koordinatensystem. \\ 
\renewcommand{\arraystretch}{1.0} 
\begin{tabular}{c|c|c|c|c|c|c}
-3 & -2 & -1 & 0 & 1 & 2 & 3\\ \hline 
15. & 9.0 & 5.0 & 3.0 & 3.0 & 5.0 & 9.0\\ 

\end{tabular} 

\definecolor{cqcqcq}{rgb}{0.75,0.75,0.75} 
\begin{tikzpicture}[domain=-3:3,scale=1][line cap=round,line join=round,>=triangle 45,x=1.0cm,y=1.0cm]\draw[color=cqcqcq,dash pattern=on 2pt off 2pt, xstep=1.0cm,ystep=1.0cm] (-3.0,-3.0) grid (3.0,3.0); 
\draw[->] (-3.0,0) -- (3.0,0) node[below] {\footnotesize $x$}; 
\foreach \x in {-3, -2, -1,  1, 2}
\draw[shift={(\x,0)},color=black] (0pt,2pt) -- (0pt,-2pt) node[below] {\footnotesize $\x$};\draw[->,color=black] (0,-3.0) -- (0,3.0) node[right] {\footnotesize $y$}; 
\foreach \y in {-3, -2, -1,  1, 2}
\draw[shift={(0,\y)},color=black] (2pt,0pt) -- (-2pt,0pt) node[left] {\footnotesize $\y$}; 
\draw[color=black] (0pt,-10pt) node[right] {\footnotesize $0$}; 
\begin{scope} 
\clip (-3,-3) rectangle (3,3); 
\draw[color=black] plot[smooth] function{x**2 - x + 3}; 
\draw (0.5,3.2) node[right] {$\scriptscriptstyle f$}; 
\end{scope} 
\end{tikzpicture}\end{aufgabe} 
\begin{aufgabe}\framebox{\qquad/4} ~ \\ 
Klaus hat sich einen neuen MP3-Player gekauft. Der MP3-Player besitzt eine Speicherkapazit\"at von $6\mathrm{\,GB}$. Wieviele Titel kann er auf den Player laden, wenn ein Musikst\"uck im Schnitt $2\mathrm{\,MB}$ belegt? 
$\frac{6.0\mathrm{\,GB}}{2.0\mathrm{\,MB}} = \frac{6.0\cdot 10^9\mathrm{\,B}}{2.0\cdot 10^6\mathrm{\,B}}=3000.0\mathrm{\,}$\end{aufgabe} 
\begin{aufgabe}\framebox{\qquad/4} ~ \\ 
Bei einem Frachterungl\"uck sind $200000\mathrm{\,l}$ \"Ol ins Meer gelaufen. Die Dicke der \"Olschicht betr\"agt $0.01\mathrm{\,mm}$. Berechne die Wasserfl\"ache, die von \"Ol bedeckt ist.
$\frac{200000.0\mathrm{\,l}}{0.01\mathrm{\,mm}} = \frac{200000.0\cdot 10^{-3}\mathrm{\,m}}{0.01\cdot 10^-3\mathrm{\,m}}=20000000.0\mathrm{\,m}^2$\end{aufgabe} 
\begin{aufgabe}\framebox{\qquad/4} ~ \\ 
Welche Seitenl\"ange hat ein $20\mathrm{\,ha}$ gro\ss{}er quadratischer Acker?
$\sqrt{20.0\mathrm{\,ha}}=\sqrt{20.0\cdot 10^4\mathrm{\,m}^2}=447.213595499958\mathrm{\,m}$\end{aufgabe} 
\begin{center} \begin{framed} Erh\"ohtes Niveau (4 Punkte) \end{framed} \end{center}\begin{aufgabe}\framebox{\qquad/4} ~ \\ 
Bringe die Funktionsgleichungen auf Scheitelpunktform. \\ 
\begin{multicols}{3} 
\begin{enumerate}[a)] 
\item 
$f(x)=x^{2} - 4 x + 3$ \\ 
$f(x)=\left(x - 2\right)^{2} - 1$ \\ 

\end{enumerate} 
\end{multicols} 
\end{aufgabe} 
\end{flushleft} 
    \end{document}