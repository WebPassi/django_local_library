\documentclass[12pt,fleqn]{article}
\usepackage[utf8]{inputenc}
\usepackage{paralist} 
\usepackage{amssymb}  
\usepackage{amsthm}   
\usepackage{eurosym} 
\usepackage{multicol} 
\usepackage{framed}
\usepackage[left=1.5cm,right=1.5cm,top=1.5cm,bottom=0.5cm]{geometry} 
\usepackage{fancyheadings} 
\pagestyle{fancy} 
\headheight1.6cm 
\lhead{Name:} 
\chead{\hspace{5cm} Klasse: \qquad\qquad Datum: \qquad\qquad} 
\rhead{\includegraphics[scale=0.3]{/home/pfranz/hbo/logo.png}} 
\lfoot{} 
\cfoot{} 
\rfoot{} 
\usepackage{amsmath}  
\usepackage{cancel} 
\usepackage{pgf,tikz} 
\usetikzlibrary{arrows} 
\newtheoremstyle{aufg} 
{16pt}  % Platz zwischen Kopf und voherigem Text 
{16pt}  % und nachfolgendem Text 
{}     % Schriftart des Koerpers 
{}     % mit \parindent Einzug 
{\bf}  % Schriftart des Kopfes 
{:}     % Nach Bedarf z.B. Doppelpunkt nach dem Kopf 
{0.5em} % Platz zwischen Kopf und Koerper 
{}     % Kopfname 

\theoremstyle{aufg} 
\newtheorem{aufgabe}{Aufgabe} 



\newtheoremstyle{bsp} 
{16pt}  % Platz zwischen Kopf und voherigem Text 
{16pt}  % und nachfolgendem Text 
{}     % Schriftart des Koerpers 
{}     % mit \parindent Einzug 
{\em}  % Schriftart des Kopfes 
{:}     % Nach Bedarf z.B. Doppelpunkt nach dem Kopf 
{0.5em} % Platz zwischen Kopf und Koerper 
{}     % Kopfname 

\theoremstyle{bsp} 
\newtheorem{beispiel}{Beispiel} 


\begin{document} 
    \begin{flushleft}
\begin{center}Lineare Funktionen\end{center} 04.09.15 \\[2em]Gegeben ist die Funktion f mit\[f(x)=\frac{x}{2} - 1\; . \]\begin{enumerate}[a)] 
\item 
Bestimmen Sie die Nullstellen von $f$. \\ 

\item 
Zeichnen Sie den Graphen von $f$. \\ 

\item 
Bestimmen Sie den Funktionswert an der Stelle $x=2$. \\ 

\item 
Bestimmen Sie, an welcher Stelle die Funktion den Wert $y=-3$ annimmt. \\ 

\item 
Untersuchen Sie die Steigung von $f$ sowohl qualitativ (fallend/steigend) als auch quantitativ. Geben Sie hierzu auch die Steigung in Prozent und den Steigungswinkel an. \\ 

\item 
Gegeben ist eine weitere Funktion $g$, deren Graph durch die Punkte $A(-5|3)$ und $B(3|-5)$ verl\"auft. Bestimmen Sie die Funktionsgleichung von $g$. \\ 

\item 
Untersuchen Sie, ob sich $f$ und $g$ schneiden und bestimmen Sie gegebenenfalls den Schnittpunkt. \\ 

\item 
Bestimmen Sie den Schnittwinkel zwischen $f$ und $g$. \\ 

\end{enumerate} 
\end{flushleft} 
\end{document}